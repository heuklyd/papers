\documentclass[a4paper,12pt]{article}

\usepackage[utf8]{inputenc}
\usepackage[width=17cm,height=23cm]{geometry}
\usepackage{amsmath,amsfonts,amssymb,amsthm}
\usepackage{mathrsfs}
\usepackage[hidelinks,pdfusetitle]{hyperref}
\usepackage[capitalise]{cleveref}
\usepackage{bbm}

\newtheorem{proposition}{Proposition}[section]
\newtheorem{theorem}[proposition]{Theorem}
\newtheorem{corollary}[proposition]{Corollary}
\newtheorem{lemma}[proposition]{Lemma}
\newtheorem{definition}[proposition]{Definition}
\newtheorem{open}[proposition]{Open problem}
\newtheorem{remark}[proposition]{Remark}
\newtheorem{example}[proposition]{Example}

% Used to rename "Problem A"
\newtheorem{innerproblem}{Problem}
\newenvironment{problem}[1]
{\renewcommand\theinnerproblem{#1}\innerproblem}
{\endinnerproblem}

\crefname{equation}{eq.}{eqs.}
\crefname{section}{Sec.}{Secs.}
\crefname{theorem}{Th.}{Ths.}
\crefname{problem}{Pb}{Pbs}

\title{A formula for the solution of a boundary value problem for the stationary equation of Brownian motion}
\author{Ju. P. \textsc{Gor'kov}\footnote{\emph{AMS (MOS) subject classifications (1970)}. Primary 35H05, 60J65; Secondary 35A22}}
\date{}

\DeclareMathOperator{\Ker}{Ker}

\newcommand{\dd}{\,\mathrm{d}}

\begin{document}

\noindent
Dokl.\ Akad.\ Nauk SSSR    
\hspace{9cm}
Soviet Math.\ Dokl.\

\noindent
Tom 223 (1975), No.\ 3 
\hspace{9cm}
Vol.\ 16 (1975), No.\ 4

{\let\newpage\relax\maketitle}

\maketitle


The present note is devoted to the solution of the following boundary value problem: find a function $u(x, y)$, continuous and bounded in the region $E[x \leq 0,-\infty<y<\infty]$, satisfying for $x<0$ the equation
\begin{equation} \label{eq:1}
L u \equiv u_{y y}-y u_{x}=f(x, y)
\end{equation}
and coinciding with the function $\phi(y)$ for $x=0$, $y \leq 0$.

The equation \eqref{eq:1} occurs in the study of Brownian motion, and it has been discussed in the papers \cite{zbMATH02540748,uhlenbeck1930theory,chandrasekhar1943stochastic} in connection with various mathematical problems. Let us also remark that the equation \eqref{eq:1} belongs to the class of hypoelliptic second-order equations; boundary value problems for equations of this type have been studied in the papers \cite{zbMATH03345727,zbMATH03280571}.

Suppose the following conditions are satisfied:
\begin{enumerate}
	\item $\sup _{x, y}|f(x, y)|<\infty$, $(x, y) \in E$;
	\item $f(x, y) \in C^{\alpha,\beta}(E)$, $1 / 3<\alpha$, $0<\beta$;
	\item $\int_{-\infty}^{0} \int_{-\infty}^{\infty}|f(x, y)|\left(x^{2}+y^{2}\right) d x d y<\infty$;
	\item $\phi(y)$ is continuous and $|\phi(y)|<C_{0}\left(1+|y|^{\delta}\right)$, where $C_{0}>0$, $0 \leq \delta <1 / 2$.
\end{enumerate}
Let
$$
\Phi(x, y, \xi, \eta)=\frac{\sqrt{3}}{2 \pi} \int_{0}^{\infty} \frac{1}{t^{2}} \exp \left\{-\frac{(y-\eta)^{2}}{4 t}-\frac{3}{t^{3}}\left(x-\xi+\frac{y+\eta}{2} t\right)^{2}\right\} \dd t .
$$
Then, because the function
$$
z(x, y)=-\int_{-\infty}^{0} \int_{-\infty}^{\infty} f(\xi, \eta) \Phi(x,-y, \xi,-\eta) d \xi d \eta
$$
satisfies equation \eqref{eq:1} as well as condition 4) (for $x=0$), it follows that the solution of the indicated problem has to be found only for $f \equiv 0$.

Let us proceed to the derivation of a formula for the solution of the problem.
Suppose $u(x, y)$ satisfies \eqref{eq:1} for $f(x, y) \equiv 0$ and the boundary condition
\begin{equation} \label{eq:2}
u(0, y)=\varphi(y), \quad y \leqslant 0 .
\end{equation}
Integrating both sides of the identity $\Phi(x,y,\xi,\eta) Lu = 0$ over the region $x \leq 0$, 
 $-\infty < y<\infty$, we then get the equality
\begin{equation} \label{eq:3}
	u(\xi, \eta)+\int_{-\infty}^{\infty} y u(0, y)\Phi(0, y, \xi, \eta) \dd y=0
\end{equation}
Now let $\xi \rightarrow 0$ in this equality, denote $u(0, \eta)$ by $\mu(\eta)$, and use the relation
$$
\Phi(0, y, 0, \eta)=\frac{\sqrt{3}}{2 \pi} \frac{1}{y^{3}+y \eta+\eta^{2}}
$$
Then we get for $\mu(\eta)$ the equation
\begin{equation} \label{eq:4}
	\mu(\eta)+\frac{\sqrt{3}}{2 \pi} \int_{0}^{\infty} \frac{\tau}{\tau^{2}+\tau+1} \mu(\eta \tau) \dd \tau=f_{0}(\eta), \quad f_{0}(\eta)=-\int_{-\infty}^{0} y \varphi(y) \Phi(0, y, 0, \eta) \dd y .
\end{equation}
The solution of \eqref{eq:4} can be found in explicit form. In fact, if one applies the Mellin transform to the left-hand right-hand sides of \eqref{eq:4}, then one obtains the new equation
\begin{gather*}
	\tilde{\mu}(s)+\frac{\sqrt{3}}{2 \pi} \int_{0}^{\infty} \frac{\tau}{\tau^{2}+\tau+1}\left[\int_{0}^{\infty} \mu(\eta \tau) \eta^{s-1} \dd \eta\right] \dd \tau=\tilde{f}_{0}(s) \\
	\left(\tilde{\mu}(s)=\int_{0}^{\infty} \mu(\eta) \cdot \eta^{s-1} \dd \eta\right).
\end{gather*}
It follows that
$$
\tilde{\mu}(s)+\frac{\sqrt{3}}{2 \pi} \tilde{\mu}(s) \int_{0}^{\infty} \frac{\tau^{1-s}}{\tau^{2}+\tau+1} \dd \tau=\tilde{f}_{0}(s).
$$
Now let $s=1+i \gamma,-\infty<\gamma<\infty$. Then
$$
\int_{0}^{\infty} \frac{\tau^{1-s}}{\tau^{2}+\tau+1} d \tau=\frac{2 \pi}{\sqrt{3}} \frac{1}{1+2 \cosh \frac{2}{3} \pi \gamma}
$$
and
\begin{equation} \label{eq:5}
	\tilde{\mu}(s)=\tilde{f}_{0}(s)-\frac{1}{4} \tilde{f}_{0}(s) \frac{1}{\cos ^{2} \frac{1}{3}\pi(1-s)}
\end{equation}
Applying the inverse Mellin transform to both sides of \eqref{eq:5}, one gets
$$
\mu(\eta)=f_{0}(\eta)-\frac{1}{8 \pi i} \int_{1-i \infty}^{1+i \infty} f_{0}(s) \frac{\eta^{-s}}{\cos ^{2} \frac{1}{3} \pi(1-s)} \dd s
$$
If we now make use of the formula (13) of \cite{zbMATH03088518}, \S 6.1, and take into account that
$$
\frac{1}{2 \pi i} \int_{1-i \infty}^{1+i \infty} \frac{\eta^{-s}}{\cos ^{2} \frac{1}{3} \pi s} \dd s=\frac{3}{2 \pi i} \eta^{\frac{3}{2}} \int_{\frac{5}{6}-i\infty}^{\frac{5}{6} +i \infty} \frac{\eta^{-3 \gamma}}{\sin ^{2} \pi \gamma} \dd \gamma=\frac{9}{\pi^{2}} \eta^{\frac{3}{2}} \frac{\ln \eta}{\eta^{3}-1}
$$
(see (20) in \cite{zbMATH03088518}, \S 7.2), then we get
$$
\frac{1}{2 \pi i} \int_{1-i\infty}^{1+i \infty} f_{0}(s) \frac{\eta^{-s}}{\cos ^{2} \frac{1}{3} \pi(1-s)} \dd s=\frac{9}{\pi^{2}} \int_{0}^{\infty} f_{0}(\eta \rho) \rho^{\frac{3}{2}} \frac{\ln \rho}{\rho^{3}-1} \dd \rho
$$
Hence
\begin{equation}
	\label{eq:6}
	\mu(\eta)=f_{0}(\eta)-\frac{9}{4\pi^{2}} \int_{0}^{\infty} f_{0}(\eta \rho) \frac{\rho^{\frac{3}{2}} \ln \rho}{\rho^{3}-1} \dd \rho
\end{equation}
The expression \eqref{eq:6} can be written in a simpler form if one makes use of the equality 
$$
\int_0^{\infty} \frac{1}{y^2 + y \eta \rho + \eta^2 \rho^2}
\frac{\rho^{\frac{3}{2}} \ln \rho}{\rho^{3}-1} \dd \rho
= 
\frac{4 \pi^2}{9} \frac{1}{y^2 + y \eta + \eta^2}
+ \frac{4\pi^2}{3 \sqrt{3}} \frac{|y|^{\frac{1}{2}} \eta^{\frac{1}{2}}}{y^3-\eta^3}
$$
where $f_{0}(\eta)$ has been replaced by its explicit value (see above).

Finally we have
\begin{equation}
	\label{eq:7}\mu(\eta)=\frac{3}{2 \pi} \int_{0}^{\infty} \frac{\tau^{\frac{3}{2}}}{\tau^{3}+1} \varphi(-\tau \eta) \dd \tau .
\end{equation}
The relations \eqref{eq:3} and \eqref{eq:7} lead to the following formula for the solution of the problem \eqref{eq:1}, \eqref{eq:2} (for $f(x, y)=0$):
\begin{equation}
	\label{eq:8}
	u(x, y)=\int_{-\infty}^{0} \gamma \varphi(\gamma) G(x, y, \gamma) \dd \gamma,
\end{equation}
where
\begin{equation}
	\label{eq:9}
	G(x, y, \gamma)=\frac{3}{2 \pi} \int_{0}^{\infty} \frac{\tau^{\frac{3}{2}}}{\tau^{3}+1} \Phi(0,-\tau \gamma, x, y) \dd \tau-\Phi(0, \gamma, x, y).
\end{equation}
The function $u(x, y)$ given by \eqref{eq:8} does indeed satisfy equation \eqref{eq:1}. This is an immediate consequence of the fact that $L \Phi(\xi, \eta, x, y)=0$ for $x \neq \xi$. In order to verify the formula \eqref{eq:8} it is also necessary to check that $u(x, y) \rightarrow \phi\left(y_{0}\right)$ when $x \rightarrow-0, y \rightarrow y_{0}$ ($y_{0} \leq 0$). This can be done by use of the equality $G(0, y, \gamma) \equiv 0$ for $y<0$, $\gamma<0$.

Let us now write down the formula for the solution of the problem \eqref{eq:1}, \eqref{eq:2}:
\begin{equation}
	\begin{split}
		u(x, y) & = \int_{-\infty}^{0} \gamma \varphi(\gamma) G(x, y, \gamma) \dd \gamma \\
		& -\int_{-\infty}^{0} \int_{-\infty}^{\infty} f(\xi, \eta)\left[\Phi(x, y, \xi, \eta)-\int_{-\infty}^{0} \gamma \Phi(0, \gamma, \xi, \eta) G(x, y, \gamma) \dd \gamma\right] \dd \xi \dd \eta .
	\end{split}
\end{equation}
If $f(x, y)$ is defined in the whole plane and if it satisfies the same conditions for $x \geq 0$ as it does for $x \leq 0$ (see 1)-3)), then the solution of the problem \eqref{eq:1}, \eqref{eq:2} can be continued to the right half-plane across the semiaxis $y \geq 0$. The formula for the solution in the plane cut along the negative $y$-axis, including the origin, is written in similar fashion.

Now let $f(x, y) \equiv 0$ and $\phi(y)=|y|^{\delta}$, $0<\delta<\frac{1}{2}$. One can show that in this case the solution to the problem \eqref{eq:1}, \eqref{eq:2} has the form
$$
u(x, y)=\bar{\mu}(x, y)\left(|x|^{\frac{\delta}{3}}+y^{\delta}\right),
$$
where $0<\mu_{0} \leq \bar{\mu}(x, y) \leq M_{1}$ ($x \leq 0$, $-\infty<y<\infty$). It follows easily from this fact that the problem \eqref{eq:1}, \eqref{eq:2} has a unique solution in the class of functions growing not faster than $M\left(|x|^{\delta / 3}+|y|^{\delta}\right)$.

Let us now consider the asymptotic behavior of the function $G(x, y, \gamma)$ as $x^{2}+y^2 \to \infty$ for finite $\gamma$. 
First we remark that if $\phi(y)=c_{1} / y+c_{2} / y^{2}+O\left(1 / y^{3}\right)$ as $y \rightarrow-\infty$,
then 
$$\mu(y)=-c_{1} / 2 y+c_{2} / y^{2}+O\left(1 / y^{\frac{5}{2}}\right) \quad \text{as} \quad y \rightarrow+\infty$$
(see [7]). Further, if $\phi(y)=a_{1} y+a_{2} y^{2}+a_{3} y^{3}+O\left(y^{4}\right)$ as $y \rightarrow-0$, then
$$\mu(y)=\frac{3}{2 \pi} \int_{-\infty}^{0} \frac{\varphi(\zeta)}{|\zeta|^{\frac{3}{2}}} \dd \zeta \cdot y^{\frac{1}{2}}+a_{1} y-\frac{a_{2}}{2} y^{2}+a_{3} y^{3}+O\left(y^{\frac{7}{2}}\right)$$
as $y \rightarrow+0$. 

Taking these formula into account on easily deduces that $G(x,y,\gamma)$ has the representation
$$G(x, y, \gamma)=\frac{3}{2 \pi} \int_{0}^{\infty} \frac{\Phi(0, \nu, \theta, 1)-\Phi(0,0, \theta, 1)}{\nu^{\frac{3}{2}}} \dd \nu +\frac{|\gamma|^{\frac{1}{2}}}{y^{\frac{5}{2}}}+O\left(\frac{1}{y^{4}}\right)$$
for $y>0$,$|\theta| \leq \theta_{0}$, $\theta=x / y^{3}$;
$$G(x, y, \gamma)=\frac{3}{2 \pi} \int_{0}^{\infty} \frac{\Phi(1, \nu, 0, s)-\Phi(1,0,0, s)}{\nu^{\frac{3}{2}}} \dd \nu \frac{|\gamma|^{\frac{1}{2}}}{|x|^{\frac{3}{2}}}+O\left(\frac{1}{|x|^{\frac{4}{3}}}\right)$$
for $|s| \leq s_{0}$, $s=y /|x|^{\frac{1}{3}}$;
$$G(x, y, \gamma)=\frac{3}{2 \pi} \int_{0}^{\infty} \frac{\Phi(0, \nu, r,-1)-\Phi(0,0, r,-1)}{\nu^{\frac{3}{2}}} \dd \nu \frac{|\gamma|^{\frac{1}{2}}}{|y|^{\frac{5}{2}}}+O\left(\frac{1}{y^{4}}\right)$$
for $y<0$, $|r| \leq r_{0}$, $r=x /|y|^{3}$ ; $\gamma<0$, $|\gamma| \leq \gamma_{0}$, $x \leq 0$.

To conclude we will briefly consider the results from a probabilistic point of view, confining ourselves for simplicity to a homogeneous equation $(f(x, y) \equiv 0)$. For a more natural probabilistic interpretation, it is convenient to rewrite the problem \eqref{eq:1},
\eqref{eq:2} in the form
\begin{equation}
	\label{eq:1'}
	\tag{$1^\prime$}
	\begin{split}
		L_{1} u \equiv \frac{1}{2} \frac{\partial^{2} u}{\partial y^{2}}+y \frac{\partial u}{\partial x}=0, \quad x<0, \quad-\infty<y<\infty, \\
		\left.u\right|_{y \geqslant 0, x=0}=\varphi\left(-2^{\frac{1}{3}} y\right) \equiv \varphi_{1}(y) .
	\end{split}
\end{equation}
The operator $L_{1}$ is the infinitesimal generator of a two-dimensional Markov process, which can be described by a system of stochastic differential equations:
\begin{equation} \label{eq:11}
	\begin{split}
		\dd X(t) & = Y(t) \dd t, \\
		\dd Y(t) & =\dd\omega(t),
	\end{split}
\end{equation}
where $\omega(t)$ is the standard Wiener process. The probabilistic representation of the solution of the problem \eqref{eq:1'} is easily obtained from the general theory of Markov processes. More precisely, if we denote by $\mathrm{E}_{x, y}(\cdot)$ the mathematical expectation that corresponds to the solution of the system \eqref{eq:11} with the initial condition $X(0)=x$, $Y(0)=y$, then (see \cite{zbMATH03215021})
\begin{equation} \label{eq:12}
u(x, y)=\mathbf{E}_{x, y} \phi_{1}(Y(\tau))=\int_{0}^{\infty} p(x, y, \gamma) \varphi_{1}(\gamma) \dd \gamma
\end{equation}
where $p(x, y, \gamma) \Delta \gamma =P_{x, y}\{Y(\tau) \in[\gamma, \gamma+\Delta \gamma]\}+o(\Delta \gamma)$ as $\Delta \gamma \rightarrow 0$, and where $\tau$ is the moment of first passage of the process \eqref{eq:11} on the half-line $y>0$. It follows from the representation \eqref{eq:8} and \eqref{eq:12} that
$$
p(x, y, \gamma)=-4^{\frac{1}{3}} \gamma G\left(x,-2^{\frac{1}{3}} y,-2^{\frac{1}{3}} \gamma\right) .
$$
I express my gratitude to A.\ M.\ Il'in for his interest in my work and his valuable remarks.

\bigskip

Institute of Mathematics and Mechanics

Ural Scientific Center 
\hspace{8cm}
received 25/DEC/74

Academy of Sciences of the USSR

\begin{thebibliography}{1}
	
	\bibitem{zbMATH02540748}
	A.~{Kolmogoroff}.
	\newblock {Zuf\"allige Bewegungen.}
	\newblock {\em {Ann. Math. (2)}}, 35:116--117, 1934.
	
	\bibitem{uhlenbeck1930theory}
	G.~Uhlenbeck and L.~Ornstein.
	\newblock On the theory of the brownian motion.
	\newblock {\em Physical review}, 36(5):823, 1930.
	
	\bibitem{chandrasekhar1943stochastic}
	S.~Chandrasekhar.
	\newblock Stochastic problems in physics and astronomy.
	\newblock {\em Reviews of modern physics}, 15(1):1, 1943.
	
	\bibitem{zbMATH03345727}
	O.~A. {Ole\u{\i}nik} and E.~V. {Radkevich}.
	\newblock {Second order equations with nonnegative characteristic form}.
	\newblock {Itogi Nauki, Ser. Mat., Mat. Anal. 1969, 7-252.}, 1971.
	
	\bibitem{zbMATH03280571}
	M.~{Derridj}.
	\newblock {Sur un probl\`eme aux limites non elliptique}.
	\newblock {\em {C. R. Acad. Sci., Paris, S\'er. A}}, 269:11--13, 1969.
	
	\bibitem{zbMATH03215021}
	E.~B. {Dynkin}.
	\newblock {\em {Markov processes. Vols. I, II. Translated.}}, volume 121/122.
	\newblock Springer, Cham, 1965.
	
	\bibitem{zbMATH03088518}
	A.~{Erd\'elyi}, W.~{Magnus}, F.~{Oberhettinger}, and F.G. {Tricomi}.
	\newblock {Tables of integral transforms. Vol. I}.
	\newblock {Bateman Manuscript Project. California Institute of Technology. New
		York: McGraw-Hill Book Co. XX, 391 p.}, 1954.
\end{thebibliography}

\end{document}

