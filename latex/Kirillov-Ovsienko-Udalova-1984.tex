\documentclass[a4paper]{article}

\usepackage[utf8]{inputenc}
\usepackage{amsmath,amsfonts,amssymb,amsthm}
\usepackage{mathrsfs}
\usepackage[hidelinks,pdfusetitle]{hyperref}
\usepackage[capitalise]{cleveref}
\usepackage{quiver} 
\usepackage{enumitem}

\newtheorem{theorem}{Theorem}
\newtheorem{lemma}{Lemma}
\newtheorem{proposition}{Proposition}
\newtheorem*{corollary*}{Corollary}

\title{Identities in the Lie Algebra of Vector Fields \\ on the Real Line\thanks{Originally published as Keldysh Inst. Prikl. Mat. Preprint No. 135. USSR Academy of Sciences, 1984. Translated by Robert M. Silvermann.}}
\date{}
\author{A. A. Kirillov, V. Yu. Ovsienko, and O. D. Udalova}

\newcommand{\C}{\mathbb{C}}
\newcommand{\N}{\mathbb{N}}
\newcommand{\R}{\mathbb{R}}
\newcommand{\ad}{\operatorname{ad}}
\newcommand{\Alt}{\operatorname{Alt}}
\newcommand{\Der}{\operatorname{Der}}
\newcommand{\End}{\operatorname{End}}
\newcommand{\Ker}{\operatorname{Ker}}
\newcommand{\Vect}{\operatorname{Vect}}

\begin{document}

{\small
\noindent
Selecta Mathematica Sovietica   
\hspace{4.2cm}
0272-9903/91/010007-11

\noindent
Vol.\ 10, No.\ 1 (1991)
\hspace{4.5cm}
\copyright 1991 Birkhäuser Verlag, Basel}
	
{\let\newpage\relax\maketitle}
	
\maketitle
	
\begin{abstract}
	In the present study, it is proved that many identities that are satisfied in the Lie algebra of vector fields on the real line are a consequence of just one of these identities: $T_4 \equiv 0$. 
	The article discusses the hypothesis that the algebra $\mathfrak{h}_k$ general by $k$ vector fields on the real line is isomorphic to a universal $T_4$-algebra with $k$ generators $\mathfrak{G}_k$.
\end{abstract}

\section*{Introduction}

The concept of the functional dimension of a representation is well known in the theory of finite-dimensional representations of Lie groups (cf.\  \cite{1,2}).
This concept is less self-evident for the case in which the Lie group is itself infinite-dimensional, and there are a number of different approaches to a definition (cf.\ \cite{3}).
One approach is associated with the algebraic structure of Lie algebras of vector fields on finite-dimensional manifolds.
It was discovered comparatively recently that nontrivial identities are satisfied in these Lie algebras (cf.\ \cite{4} and the literature cited there).
That is, suppose that
\begin{equation}
	\label{eq:1}
	T_k = T_k(x_1,x_2, \dotsc, x_k ; y) = \Alt(\ad x_1 \dotsb \ad x_k) y
\end{equation}
where $\Alt$ denotes the alternating sum over all permutations of the variables.

Simple calculations (cf., for example, \cite{4}) show that the identity $T_k \equiv 0$ is satisfied in the algebra $\Vect(M^n)$ of vector fields on an $n$-dimensional manifold $M^n$ when $k \geq (n+1)^2$.

Correspondingly, the identity $T_4 \equiv 0$ is satisfied in the Lie algebra $\Vect(\R^1)$ of vector fields on the real line.
Note that if $M$ is an arbitrary manifold (possibly infinite-dimensional) and if $P$ is a one-dimensional bundle on $M$, all the identities of the Lie algebra $\Vect(\R^1)$, in particular, the identity $T_4 \equiv 0$, are satisfied in the Lie algebra $\Vect(M,P)$ of all vector fields on $M$ that are tangent to the fibers of $P$.

Let us consider the category $A_k$ of all Lie algebras with $k$ generators that may be realized as subalgebras in $\Vect(M,P)$, where $P$ is a one-dimensional bundle.
(Morphisms in this category are homomorphisms of the Lie algebra that carries generators into generators.)
Let us prove that there exists a universal object $\mathfrak{h}_k$ in $A_k$.
We introduce a countable number of independent variables $\varphi_i^j$, $1 \leq j \leq k-1$, $i = 0, 1, 2, \dotsc$, and define the operator $\mathscr{D}$ in $\C[\varphi_i^j]$ by the formula
\begin{equation}
	\label{eq:2}
	\mathscr{D} = \sum_{j=1}^{k-1} \sum_{i=1}^\infty \varphi_i^j \frac{\partial}{\partial \varphi_{i-1}^j}.
\end{equation}

The operator $\mathscr{D}$ may be interpreted as a linear vector field on an infinite-dimensional space $M$ with coordinates $\varphi_i^j$.
Suppose that $P$ is a one-dimensional bundle generated by the field $\mathscr{D}$.
We denote by $\mathfrak{h}_k$ the Lie algebra generated by the $k$ vector fields
\begin{equation*}
	\mathscr{D}, \varphi_0^1 \mathscr{D}, \dotsc, \varphi_0^{k-1} \mathscr{D}.
\end{equation*}

By construction, $\mathfrak{h}_k$ is an object of the category $A_k$.
If $\mathfrak{h}$ is another object of $A_k$ generated by the fields
\begin{equation*}
	\xi_0,\xi_1,\dotsc,\xi_{k-1}\in \Vect(\tilde{M},\tilde{P}),   
\end{equation*}
then there will exist a homomorphism $\tau : \mathfrak{h}_k \to \mathfrak{h}$ that carries $\mathscr{D}$ into $\xi_0$ and $\varphi_0^i \mathscr{D}$ into $\xi_i$, $1 \leq i \leq k-1$.
In fact, we need only verify that for any Lie polynomial $L(x_1,\dotsc,x_k)$, it follows from the equality $L(\mathscr{D},\varphi_0^1 \mathscr{D}, \dotsc, \varphi_0^{k-1}\mathscr{D}) = 0$ that $L(\xi_0,\dotsc,\xi_{k-1}) = 0$.
It may be assumed that, in appropriate coordinates $(x_0,x_1,\dotsc)$ on $\tilde{M}$, the bundle $\tilde{P}$ is locally rectified and is generated by the field $\xi_0 = \partial/{\partial x_0}$.
(We will not go into the details of the theory of vector fields on infinite-dimensional manifolds, since this is not essential to the purely algebraic discussions that follow.)
The fields $\xi_i$ assume the form $\xi_i = y_i \partial / \partial x_0$, where $y_i$ are certain smooth functions on $\tilde{M}$.
The field $L(\xi_0,\dotsc, \xi_{k-1}) = 0$ may be written in the form
\begin{equation*}
	P\left(y_1,\dotsc,y_{k-1};\frac{\partial y_1}{\partial x_0}, \dotsc, \frac{\partial y_{k-1}}{\partial x_0}, \dotsc, \frac{\partial^m y_{k-1}}{\partial x_0^m} \right) \frac{\partial}{\partial x_0},
\end{equation*}
where $P$ is a polynomial.
At the same time, $L(\mathscr{D},\varphi_0^1\mathscr{D}, \dotsc, \varphi_0^{k-1}\mathscr{D})$ may be written in the form
\begin{equation*}
	P(\varphi_0^1, \dotsc, \varphi_0^{k-1}, \varphi_1^1, \dotsc, \varphi_1^{k-1}, \dotsc, \varphi_m^{k-1}) \mathscr{D}.
\end{equation*}
By definition, this expression must equal $0$, whence $P \equiv 0$, and the assertion is proved.
Thereby we have also proved the universality of $\mathfrak{h}_k$.

Now suppose that $\mathscr{B}_k$ is the category of Lie algebras with $k$ generators in which the identity $T_4 \equiv 0$ is satisfied.
We will call the objects of $\mathscr{B}_k$ Lie $T_4$-algebras.
There is also a universal object in this category, which we denote by $\mathfrak{G}_k$.
It is the quotient algebra of a free algebra with $k$ generators in each ideal generated by elements of the form $T_4(x_1,x_2,x_3,x_4;y)$.

The algebras $\mathfrak{h}_k$ and $\mathfrak{G}_k$ share a number of features (cf.\ \cref{thm:1,thm:2,thm:3,thm:4} below), which lead us to the hypothesis that they might be isomorphic.
More precisely, we suppose that the canonical epimorphism $\pi : \mathfrak{G}_k \to \mathfrak{h}_k$, which exists by virtue of the definition of $\mathfrak{G}_k$, is in fact an isomorphism.

The algebras of vector fields on manifolds are differentiation algebras of the commutative associative algebras of functions on these manifolds.
We begin our study of $\mathfrak{G}_k$ by constructing for it a commutative associative algebra $A(\mathfrak{G}_k)\subset \End(\mathfrak{G}_k)$ (by analogy with the algebras of vector fields) such that $\mathfrak{G}_k$ acts on the algebra $A(\mathfrak{G}_k)$ by means of differentiations, and moreover, the mapping $\mathfrak{G}_k \to \Der A(\mathfrak{G}_k)$ specified by this action is a morphism of $A(\mathfrak{G}_k)$-modules.

In \cref{sec:2} we present a ``free algebra theorem'' for the Lie algebras $\mathfrak{h}_k$, which may be considered an analog of the well-known Shirshov theorem \cite{6} on the subalgebras of a free Lie algebra.

Proofs of these results are given below in \cref{sec:3}.

\section{Statement of fundamental results}
\label{sec:1}

The following fact will guide our discussion in the construction of the algebra $A(\mathfrak{G}_k)$.

If $x_i = x_i(t) \mathrm{d}/\mathrm{d} t$, $y = y(t) \mathrm{d}/\mathrm{d} t$ are elements of $\Vect (\R^1)$, then it may be verified directly that 
\begin{equation}
	\label{eq:3}
	T_3(x_1,x_2,x_3;y) = - 2
	\left|
	\begin{matrix}
		x_1 & x_2 & x_3 \\
		x_1' & x_2' & x_3' \\
		x_1'' & x_2'' & x_3'' \\
	\end{matrix}
	\right|
	\cdot y.
\end{equation}

This means that the operator
\begin{equation*}
	a(x_1,x_2,x_3) = \Alt(\ad x_1 \ad x_2 \ad x_3)
\end{equation*}
is an operator for multiplication by a function in $\Vect(\R^1)$.

\begin{theorem}
	\label{thm:1}
	For any Lie $T_4$-algebra $\mathfrak{G}$, the operators $a(x_1,x_2,x_3)$, $x_i \in \mathfrak{G}$, generate a commutative subalgebra $A(\mathfrak{G})$ in $\End(\mathfrak{G})$.
\end{theorem}

The following lemma reveals to some extent the structure of the subalgebra $A(\mathfrak{G})$ in the case $\mathfrak{G} = \mathfrak{h}_k$.

\begin{lemma}
	\label{lem:1}
	The quotient field $B(\mathfrak{h}_k)$ of the algebra $A(\mathfrak{h}_k)$ consists of all rational functions of $\varphi_i^j$, $1 \leq j \leq k-1$, $i = 0, 1, \dotsc$.
\end{lemma}

\begin{proof}
	Suppose that $a \in A(\mathfrak{h}_k)$ is such that $\mathscr{D}(a) \neq 0$ (for example, $a = a(\mathscr{D},\varphi_0^1\mathscr{D},\varphi_1^1\mathscr{D})$).
	Then 
	\begin{equation*}
		(\varphi_i^j \mathscr{D})(a) = \varphi_i^j (\mathscr{D}(a)) \in A(\mathfrak{h}_k).
	\end{equation*}
	Hence
	\begin{equation*}
		\varphi_i^j = (\varphi_i^j \mathscr{D}(a)) (\mathscr{D}(a))^{-1} \in B(\mathfrak{h}_k).
		\qedhere
	\end{equation*}
\end{proof}

\begin{theorem}
	\label{thm:2}
	Suppose that $\mathfrak{G}$ is a Lie $T_4$-algebra and that $\xi \in \mathfrak{G}$ and $a, a_1, a_2 \in A(\mathfrak{G})$.
	Then:
	\begin{enumerate}
		\item \label{it:1} $[\ad \xi, a] \in A(\mathfrak{G})$,
		\item \label{it:2} $[\ad \xi, a_1 a_2] = [\ad \xi, a_1] a_2 + a_1[\ad \xi, a_2]$.
	\end{enumerate}
\end{theorem}

Thus, the action of $\mathfrak{G}$ on $A(\mathfrak{G})$ may be determined by the formula
\begin{equation}
	\label{eq:4}
	\xi(a) = [\ad \xi, a]
\end{equation}
and the action operators will constitute differentiations of the algebra $A(\mathfrak{G})$.
In the case $\mathfrak{G} = \Vect (\R^1)$, this action coincides with the ordinary action of vector fields on functions.

Suppose that $A$ is an arbitrary commutative algebra, and that $\Der A$ is the Lie differentiation algebra of $A$.
Clearly, $\Der A$ is an $A$-module relative to the operation
\begin{equation}
	\label{eq:5}
	(a\xi)(b) = a \cdot \xi(b).
\end{equation}
We suppose that $\mathfrak{p} \subset \Der A$ is simultaneously a Lie subalgebra and an $A$-submodule.
We will call $\mathfrak{p}$ an $A$-subalgebra.
Then for $\xi,\eta \in \mathfrak{p}$, $a,b \in A$, we have the following equalities:
\begin{gather}
	\label{eq:6}
	\xi(a) \eta = [\ad \xi,a]\eta, \\
	\label{eq:7}
	[a \xi, b \eta] = a \xi(b) \eta - b \eta(a) \xi + ab [\xi, \eta].
\end{gather}
Indeed,
\begin{equation*}
	\begin{split}
		[\ad \xi, a]\eta 
		& = \ad \xi (a\eta) - a \ad \xi(\eta)
		= [\xi, a\eta] - a[\xi,\eta] 
		= \xi a \eta - a \eta \xi - a[\xi,\eta] \\
		& = \xi(a) \eta + a \xi \eta - a \eta \xi - a [\xi,\eta] = \xi(a) \eta,
	\end{split}
\end{equation*}
and \eqref{eq:7} follows from \eqref{eq:6}:
\begin{equation*}
	[\xi,b\eta] = \ad \xi (b\eta)
	= [\ad \xi,b] \eta + b \ad \xi(\eta)
	= \xi(b) \eta + b[\xi,\eta].
\end{equation*}

\begin{theorem}
	\label{thm:3}
	For any Lie $T_4$-algebra $\mathfrak{G}$, the mapping of $\mathfrak{G}$ into $\Der A(\mathfrak{G})$ specified by \eqref{eq:4} is a morphism of $A(\mathfrak{G})$-modules, so that its image $\mathfrak{p}$ is an $A(\mathfrak{G})$-subalgebra in $\Der A(\mathfrak{G})$. 
\end{theorem}

As a module over an algebra of functions, the Lie algebra $\Vect(M,P)$ possesses yet another property.
That is, any two elements of $\Vect(M,P)$ are linearly dependent over an algebra of functions.

\begin{theorem}
	\label{thm:4}
	Suppose that $\mathfrak{G}$ is a Lie $T_4$-algebra.
	For any $\xi,\eta \in \mathfrak{G}$ and any $a \in A(\mathfrak{G})$,
	\begin{equation}
		\label{eq:8}
		\xi(a) \eta = \eta(a) \xi.
	\end{equation}
\end{theorem}

Note that \eqref{eq:8} is self-evident for $\mathfrak{h}_k$:
\begin{equation*}
	P(\varphi) \mathscr{D}(a) Q(\varphi) \mathscr{D} = 
	Q(\varphi) \mathscr{D}(a) P(\varphi) \mathscr{D}.
\end{equation*}

\section{Free algebra theorem}
\label{sec:2}

The fundamental result of the present section is the following assertion.

\begin{theorem}
	\label{thm:5}
	Any subalgebra in $\mathfrak{h}_k$ generated by two linearly independent elements is isomorphic to $\mathfrak{h}_2$.
\end{theorem}

\begin{proof}
	Suppose that the subalgebra $\mathfrak{h} \subset \mathfrak{h}_k$ is generated by the elements $P(\varphi) \mathscr{D}$ and $Q(\varphi) \mathscr{D}$, where
	\begin{equation*}
		\varphi = (\varphi_0^1, \dotsc, \varphi_0^{k-1}, \dotsc ; \dotsc \varphi_N^{k-1}),
	\end{equation*}
	and that the polynomials $P$ and $Q$ are linearly independent. 
	Our goal is to prove that the mapping $\tau : \mathfrak{h}_2 \to \mathfrak{h}$, which exists by virtue of the universality of~$\mathfrak{h}_2$, possesses the property that $\Ker \tau = \{ 0 \}$.
	Let us represent $\tau$ in the form of a composition of two mappings.
	For this purpose, we consider the Lie algebra $\mathfrak{G}$ of expressions of the form $R(\varphi) \mathscr{D}$, where $R$ is a rational function with a natural law of commutativity:
	\begin{equation*}
		[R_1(\varphi) \mathscr{D}, R_2(\varphi)\mathscr{D}] = (R_1 \mathscr{D} R_2 - R_2 \mathscr{D} R_1) \mathscr{D}.
	\end{equation*}
	Suppose that $\tau_1$ is a homomorphism of $\mathfrak{h}_2$ into $\mathfrak{G}$ that carries $\mathscr{D}$ into $\mathscr{D}$ and $\varphi_0^1 \mathscr{D}$ into $S(\varphi) \mathscr{D}$, where $S(\varphi)$ is a not constant function.
	
	\begin{lemma}
		\label{lem:2}
		$\Ker \tau_1 = \{ 0 \}$.
	\end{lemma}

	\begin{proof}
		Suppose that $L(x_1,x_2)$ is a Lie polynomial that does not vanish under the substitution $x_1 = \mathscr{D}$, $x_2 = \varphi_0^1 \mathscr{D}$.
		Then
		\begin{equation*}
			L(\mathscr{D}, \varphi_0^1 \mathscr{D}) = P_L(\varphi_0^1, \varphi_1^1, \dots, \varphi_N^1) \mathscr{D},
		\end{equation*}
		where $P_L$ is some polynomial. 
		Therefore,
		\begin{equation*}
			L(\mathscr{D}, S(\varphi)\mathscr{D}) = P_L(\psi_0,\dotsc,\psi_N) \mathscr{D},
		\end{equation*}
		where $\psi_k = \mathscr{D}^k S$.
		
		It remains for us to check the algebraic independence of the variables $\psi_0, \dotsc,$ $\psi_N$.
		For this purpose, we arrange the variables $\varphi_i^j$ so that the effect of $\mathscr{D}$ is to increase the ordinal number of the variable. 
		(Lexicographic ordering, for example, would be appropriate.)
		Suppose that $n(S)$ is the greatest ordinal number such that $\partial S / \partial \varphi_n \neq 0$.
		Then
		\begin{equation*}
			n(\psi_0) = n(S) < n(\psi_1) < \dotsb < n(\psi_N). \qedhere
		\end{equation*}
	\end{proof}

	Now let us consider the homomorphism $\tau_2$ of the algebra $\tau_1(\mathfrak{h}_2)$ into $\mathfrak{G}$ that carries elements of $\mathscr{D}$ and $S\mathscr{D}$ into $P\mathscr{D}$ and $PS\mathscr{D}$, respectively.
	The existence of $\tau_2$ is guaranteed by \cref{lem:2} and the universality of $\mathfrak{h}_2$.
	
	\begin{lemma}
		\label{lem:3}
		If $P \neq 0$, then $\Ker \tau_2 = \{ 0 \}$.
	\end{lemma}
	
	\begin{proof}
		The proof of the lemma is similar to the preceding proof.
		If
		\begin{equation*}
			L(\mathscr{D},S\mathscr{D}) = P_L(\psi_0,\dotsc,\psi_N) \mathscr{D},
		\end{equation*}
		then
		\begin{equation*}
			L(P\mathscr{D},PS\mathscr{D}) = P_L(\theta_0,\dotsc,\theta_N) P \mathscr{D},
		\end{equation*}
		where $\theta_k = (P(\varphi)\mathscr{D})^k S$.
		The algebraic independence of the variables $\theta_0, \dotsc, \theta_N$ may be verified as above.
		The lemma is proved.
	\end{proof}

	The assertion of the theorem follows from \cref{lem:2,lem:3} if we set $S = Q/P$.	
\end{proof}

By means of \cref{thm:1,thm:2,thm:3,thm:4,thm:5}, we may find numerous subalgebras in $\mathfrak{G}_k$ that are isomorphic to $\mathfrak{h}_2$.

\begin{lemma}
	\label{lem:4}
	If $a \in A(\mathfrak{G}_k)$ so that $\pi(a) \neq 0$, then the subalgebra $\mathfrak{b}_{a,i}$ in $\mathfrak{G}_k$ generated by $\xi_i$ and $a_{\xi_i}$ is isomorphic to the Lie algebra $\mathfrak{h}_2$.
\end{lemma}

\begin{proof}
	Under the canonical homomorphism $\pi : \mathfrak{G}_k \to \mathfrak{h}_k$, 
	% https://q.uiver.app/#q=WzAsMyxbMSwwLCJcXG1hdGhmcmFre2J9X3thLGl9Il0sWzIsMSwiXFxwaShcXG1hdGhmcmFre2J9X3thLGl9KSJdLFswLDEsIlxcbWF0aGZyYWt7aH1fMiJdLFswLDEsIlxccGkiXSxbMiwwLCJcXHRhdSJdLFsxLDIsIlxcc2lnbWEiLDAseyJzdHlsZSI6eyJ0YWlsIjp7Im5hbWUiOiJhcnJvd2hlYWQifX19XV0=
	\[\begin{tikzcd}
		& {\mathfrak{b}_{a,i}} \\
		{\mathfrak{h}_2} && {\pi(\mathfrak{b}_{a,i})}
		\arrow["\pi", from=1-2, to=2-3]
		\arrow["\tau", from=2-1, to=1-2]
		\arrow["\sigma", tail reversed, from=2-3, to=2-1]
	\end{tikzcd}\]
	the algebra $\mathfrak{b}_{a,i}$ is mapped onto the algebra $\pi(\mathfrak{b}_{a,i})$ that is generated by the linearly independent elements $\varphi_0^{i-1}\mathscr{D}$, $\pi(a)\varphi_0^{i-1}\mathscr{D}$, and consequently, by \cref{thm:5}, the Lie algebra $\pi(\mathfrak{b}_{a,i})$ is isomorphic to the Lie algebra $\mathfrak{h}_2$.
	Since $\xi_i$ is a differentiation of the algebra $A(\mathfrak{G}_k)$, and since the commutative algebra
	\begin{equation*}
		B(\mathfrak{h}_2) = \C[\varphi_0,\varphi_1,\dotsc,\varphi_N,\dotsc]
	\end{equation*}
	is mapped homomorphically into $A(\mathfrak{G}_k)$,
	\begin{equation*}
		\varphi_0 \to a, \dotsc, \varphi_n \to \xi_i^n(a),
	\end{equation*}
	so that $A(\mathfrak{h}_2) \overset{\text{onto}}{\to}A(\mathfrak{b}_{a,i})$, then the mapping of generators $\tau : \mathscr{D} \to \xi_i$, $\varphi_0 \mathscr{D} \to a \xi_i$ is continued to a homomorphism of the Lie algebra $\mathfrak{h}_2$ onto $\mathfrak{b}_{a,i}$.
	
	All three mappings $\tau$, $\pi$, and $\sigma$ carry generators into generators, and consequently $\sigma \circ \pi \circ \tau = \operatorname{id}$, and also $\tau$ is an isomorphism of the Lie algebras $\mathfrak{h}_2$ and $\mathfrak{b}_{a,i}$.
	The lemma is proved.
\end{proof}

From this lemma we obtain the following assertion.

\begin{corollary*}
	To prove the isomorphism between $\mathfrak{G}_2$ and $\mathfrak{h}_2$, it is necessary and sufficient to prove that $\mathfrak{G}_2$ is canonically isomorphic to two of its own subalgebras, namely those that are generated by $\xi_1$, $[\xi_1,\xi_2]$ and $\xi_1, [[\xi_1,\xi_2],\xi_2]$, respectively.
\end{corollary*}

The necessity follows from \cref{thm:5}; and sufficiency, from \cref{lem:4}, applied to 
\begin{equation*}
	a = a(\xi_1,\xi_2,[\xi_1,\xi_2]).
\end{equation*}

\section{Proofs of \texorpdfstring{\cref{thm:1,thm:2,thm:3,thm:4}}{Theorems 1 to 4}}
\label{sec:3}

To avoid cumbersome notation, we will denote alternation by the index of the sum of products of operators of $\ad \xi$, omitting the symbols $\Alt$ and $\ad$, i.e., the notation $\xi_{1}\xi_{2}\eta \xi_{3}$ will denote the operator $\Alt_\xi(\ad \xi_{1} \ad \xi_{2} \ad \eta \ad \xi_{3})$ and similarly
\begin{equation*}
	\xi_{1}\xi_{2}[\eta \xi_{3}](\zeta) := \Alt_\xi(\ad \xi_{1} \ad \xi_{2} \ad [\eta, \xi_{3}])(\zeta) = \Alt_\xi[\xi_{1}[\xi_{2}[[\eta \xi_{3}]\zeta]]].
\end{equation*}

To prove \cref{thm:1}, we will require an equivalent form of the identity $T_4 \equiv 0$ which is of independent interest.

\begin{proposition}
	The identity $T_4 \equiv 0$ is equivalent to
	\begin{equation}
		\label{eq:9}
		\eta \xi_1 \xi_2 \xi_3-2 \xi_1 \xi_2 \eta \xi_3 \equiv 0.
	\end{equation}
\end{proposition} 

\begin{proof}
	Writing $T_4\left(\eta, \xi_1, \xi_2, \xi_3 ; \zeta\right)=(\eta \xi_1 \xi_2 \xi_3-\xi_1 \eta \xi_2 \xi_3+\xi_1 \xi_2 \eta \xi_3-\xi_1 \xi_2 \xi_3 \eta)(\zeta)$, using Jacobi's identity and anticommutativity, we insert $\eta$ at the rightmost position, replacing $\zeta$:
	\begin{equation*}
		\begin{aligned}
			-\xi_1 \xi_2 \xi_3 \eta(\zeta) & =\xi_1 \xi_2 \xi_3 \zeta(\eta), \\
			\xi_1 \xi_2 \eta \xi_3(\zeta) & =\xi_1 \xi_2[\eta, \xi_3](\zeta)+\xi_1 \xi_2 \xi_3 \eta(\zeta) \\
			& =(\xi_1 \xi_2 \zeta \xi_3-\xi_1 \xi_2 \xi_3 \zeta)(\eta) \\
			-\xi_1 \eta \xi_2 \xi_3(\zeta) & =-\xi_1[[\eta, \xi_2], \xi_3](\zeta)-\xi_1 \xi_2 \xi_3 \eta(\zeta) \\
			& =(-\xi_1 \zeta \xi_2 \xi_3+\xi_1 \xi_2 \xi_3 \zeta)(\eta) .
		\end{aligned}
	\end{equation*}
	In this last case and in what follows, we express $\eta$ at once in terms of two variables, since $\xi_1$ and $\xi_2$ are alternating, while the other two terms are reduced:
	\begin{equation*}
		\begin{aligned}
			\eta \xi_1 \xi_2 \xi_3(\zeta) & =[[\eta, \xi_1], \xi_2] \xi_3(\zeta)+\xi_1 \xi_2 \eta \xi_3(\zeta) \\
			& =[[[\eta, \xi_1], \xi_2], \xi_3](\zeta)+\xi_3[[\eta, \xi_1], \xi_2](\zeta)+\xi_1 \xi_2 \eta \xi_3(\zeta) \\
			& =(-\zeta \xi_1 \xi_2 \xi_3+\xi_1 \zeta \xi_2 \xi_3+\xi_1 \xi_2 \zeta \xi_3-\xi_1 \xi_2 \xi_3 \zeta)(\eta) .
		\end{aligned}
	\end{equation*}
	Adding together these equalities, term by term, yields the desired formula.
\end{proof}

\begin{corollary*}
	In a Lie $T_4$-algebra, $\xi_1 \xi_2 \eta \xi_3 \xi_4=0$, since
	\begin{equation}
		\label{eq:10}
		0=\eta \xi_1 \xi_2 \xi_3 \xi_4=2 \xi_1 \xi_2 \eta \xi_3 \xi_4.
	\end{equation}
\end{corollary*}


Now let us prove \cref{thm:1}.
In accordance with our stipulations about notation, the commutator of the two operators $a(\xi_1, \xi_2, \xi_3)$ and $a(\eta_1, \eta_2, \eta_3)$ has the form
\begin{equation*}
	\xi_1 \xi_2 \xi_3 \eta_1 \eta_2 \eta_3-\eta_1 \eta_2 \eta_3 \xi_1 \xi_2 \xi_3.
\end{equation*}
Let us consider ``randomizing'' permutations of this expression, that is, operators that may be constructed by alternating the two triples of elements
\begin{equation*}
	\ad \xi_1, \ad \xi_2, \ad \xi_3 \text { and } \ad \eta_1, \ad \eta_2, \ad \eta_3
\end{equation*}
within each of which an alternation occurs.
It is easier to prove that all these operators are equal to zero simultaneously than to prove that just the first one is equal to zero.

There are precisely 10 pairs:
\begin{enumerate}[label=\Roman*., itemsep=0mm]
	\item $\xi_1\xi_2\xi_3\eta_1\eta_2\eta_3-\eta_1\eta_2\eta_3\xi_1\xi_2\xi_3$
	
	\item $\xi_1 \xi_2 \eta_1 \xi_3 \eta_2 \eta_3 - \eta_1 \eta_2 \xi_1 \eta_3 \xi_2 \xi_3$
	
	\item $\xi_1 \xi_2 \eta_1 \eta_2 \xi_3 \eta_3 - \eta_1 \eta_2 \xi_1 \xi_2 \eta_3 \xi_3$
	
	\item $\xi_1 \xi_2 \eta_1 \eta_2 \eta_3 \xi_3 - \eta_1 \eta_2 \xi_1 \xi_2 \xi_3 \eta_3$
	
	\item $\xi_1 \eta_1 \xi_2 \xi_3 \eta_2 \eta_3 - \eta_1 \xi_1 \eta_2 \eta_3 \xi_2 \xi_3$

	\item $\xi_1 \eta_1 \xi_2 \eta_2 \xi_3 \eta_3 - \eta_1 \xi_1 \eta_2 \xi_2 \eta_3 \xi_3$
	
	\item $\xi_1 \eta_1 \xi_2 \eta_2 \eta_3 \xi_3 - \eta_1 \xi_1 \eta_2 \xi_2 \xi_3 \eta_3$
	
	\item $\xi_1 \eta_1 \eta_2 \xi_2 \xi_3 \eta_3 - \eta_1 \xi_1 \xi_2 \eta_2 \eta_3 \xi_3$
	
	\item $\xi_1 \eta_1 \eta_2 \xi_2 \eta_3 \xi_3 - \eta_1 \xi_1 \xi_3 \eta_2 \xi_3 \eta_3$
	
	\item $\xi_1 \eta_1 \eta_2 \eta_3 \xi_2 \xi_3 - \eta_1\xi_1 \xi_2 \xi_3 \eta_2 \eta_3$
\end{enumerate}

Like identities \eqref{eq:9} and \eqref{eq:10}, the identity $T_4 \equiv 0$ yields a linear dependence between them:

\medskip

\begin{tabular}{p{2cm}p{8cm}}
	$\textrm{I} + \textrm{IV} = 0$
	& since $\xi_1 \xi_2 \xi_3 \eta_1 \eta_2 \eta_3 = 2 \xi_1 \xi_2 \eta_1 \eta_2 \xi_3 \eta_3$ (identity \eqref{eq:9}, written for $\eta_1, \eta_2, \eta_3, \xi_3$) = $2 \circlearrowleft_\eta \xi_1 \xi_2 [\eta_1,\eta_2] \xi_3\eta_3$ $=\eta_1 \eta_2\xi_1 \xi_2 \xi_3 \eta_3$ (identity \eqref{eq:9} for $\xi_1,\xi_2,\xi_3,[\eta_1,\eta_2]$) and analogously $\eta_1\eta_2\eta_3\xi_1\xi_2\xi_3=\xi_1\xi_2\eta_1\eta_2\eta_3\xi_3$
	\\
	$\textrm{I}-2\cdot\textrm{III} = 0$ & since $\xi_1\xi_2\xi_3\eta_1\eta_2\eta_3 = 2 \xi_1\xi_2\eta_1\eta_2\xi_3\eta_3$
	\\	
	$\textrm{I}-2\cdot\textrm{VIII} = 0$ & since $\xi_1\xi_2\xi_3\eta_1\eta_2\eta_3 = \circlearrowleft_\xi \xi_1[\xi_2,\xi_3]\eta_1\eta_2\eta_3 $ = $2 \circlearrowleft_\xi \xi_1 \eta_1 \eta_2 [\xi_2, \xi_3] \eta_3 = 2 \xi_1 \eta_1 \eta_2 \xi_2 \xi_3\eta_3$ (identity \eqref{eq:9} for $\eta_1,\eta_2,\eta_3,[\xi_2,\xi_3]$)
	\\
	$\frac 52 \cdot \textrm{I}-\textrm{II} = 0$ & since $\textrm{I}-\textrm{II}+\textrm{III}-\textrm{IV} = 0$ ($T_4(\xi_3,\eta_1,\eta_2,\eta_3;\dotsc) = 0$)
	\\
	$2\cdot\textrm{II}+\textrm{X} = 0$ & since $2\xi_1\xi_2\eta_1\xi_3\eta_2\eta_3 = \eta_1\xi_1\xi_2\xi_3\eta_2\eta_3$ (identity \eqref{eq:9} for $\xi_1,\xi_2,\xi_3,\eta_1$)
	\\
	$\frac {13}{2} \cdot \textrm{I}-\textrm{V} = 0$ & since $\textrm{I}-\textrm{II}+\textrm{V}+\textrm{X} = 0$ ($T_4(\xi_1,\xi_2,\xi_3,\eta_1;\dotsc) = 0$) 
	\\ 
	$\textrm{IV}-2\cdot\textrm{IX} = 0$ & since $\xi_1\xi_2\eta_1\eta_2\eta_3\xi_3 = 2\xi_1\eta_1\eta_2\xi_2\eta_3\xi_3$ (identity \eqref{eq:9} for $\eta_1,\eta_2,\eta_3,\xi_2$)
	\\
	$6\cdot \textrm{I}-\textrm{VI} = 0$ & since $\textrm{I}-\textrm{II}+\textrm{III}+\textrm{V}-\textrm{VI}+\textrm{VIII} = 0$
	\newline ($T_4(\xi_2,\xi_3,\eta_1,\eta_2;\dotsc) = 0$)
	\\ 
	$\frac 72 \cdot \textrm{I}-\textrm{VII} = 0$ & since $\textrm{IV}-\textrm{VII}+\textrm{IX}-\textrm{X} = 0$ ($T_4(\eta_1,\xi_2,\eta_2,\eta_3;\dotsc) = 0$)
\end{tabular}

\medskip

Identity \eqref{eq:10} yields one more linear equation independent of the preceding equations:
\begin{equation*}
	0 = \textrm{II} - \textrm{III} - \textrm{VIII} - \textrm{VII} = \left(\frac 52 - \frac 12 - \frac 12 - \frac 72\right) \textrm{I} = - 2 \textrm{I}.
\end{equation*}
Hence the assertion of \cref{thm:1} follows.

\begin{proof}[Proof of \cref{thm:2}]
	Let us prove that
	\begin{equation} 
		\label{eq:11}
		\left[\ad \xi, a\left(\eta_1, \eta_2, \eta_3\right)\right]=a\left(\left[\xi, \eta_1\right], \eta_2, \eta_3\right)+a\left(\eta_1,\left[\xi, \eta_2\right], \eta_3\right)+a\left(\eta_1, \eta_2,\left[\xi, \eta_3\right]\right).
	\end{equation}
	In fact,
	\begin{equation*}
		\begin{split}
			\xi \eta_1 \eta_2 \eta_3-\eta_1 \eta_2 \eta_3 \xi & =\xi \eta_1 \eta_2 \eta_3-\eta_1 \xi \eta_2 \eta_3+\eta_1 \xi \eta_2 \eta_3-\eta_1 \eta_2 \xi \eta_3
			\\
			& \quad \quad \quad +\eta_1 \eta_2 \xi \eta_3-\eta_1 \eta_2 \eta_3 \xi \\
			& =\left[\xi, \eta_1\right] \eta_2 \eta_3+\eta_1\left[\xi, \eta_2\right] \eta_3+\eta_1 \eta_2\left[\xi, \eta_3\right] .
		\end{split}
	\end{equation*}
	Here alternation occurs with respect to $\eta_1, \eta_2, \eta_3$. But if we group the terms containing
	\begin{equation*}
		\left[\xi, \eta_1\right], \quad\left[\xi, \eta_2\right], \quad\left[\xi, \eta_3\right]
	\end{equation*}
	we obtain \eqref{eq:11}, and consequently, Assertion \ref{it:1} of \cref{thm:2} for the generators of the algebra $A(\mathfrak{G})$.
	Assertion \ref{it:2} is a corollary of the fact that the commutator is a differentiation. 
	In our case,
	\begin{equation*}
		\begin{split}
			{\left[\ad \xi, a_1 a_2\right] } & =\ad \xi a_1 a_2-a_1 a_2 \ad \xi \\
			& =\ad \xi a_1 a_2-a_1 \ad \xi a_2+a_1 \ad \xi a_2-a_1 a_2 \ad \xi \\
			& =\left[\ad \xi, a_1\right] a_2+a_1\left[\ad \xi, a_2\right] .
		\end{split}
	\end{equation*}
	Hence Assertion \ref{it:1} follows for an arbitrary $a \in A(\mathfrak{G})$.
\end{proof}

\begin{proof}[Proof of \cref{thm:3}]
	It is necessary and sufficient to prove that Equation \eqref{eq:5} is satisfied for the Lie $T_4$-algebra $\mathfrak{G}$ and both the algebras $A(\mathfrak{G}) \subset \End \mathfrak{G}$ defined in \cref{sec:1} and the action of $\mathfrak{G}$ on $A(\mathfrak{G})$; that is, to prove that
	\begin{equation*}
		a(\xi)(b)=a \xi(b)
	\end{equation*}
	for $\xi \in \mathfrak{G}, a, b \in A(\mathfrak{G})$ under the condition that
	\begin{equation*}
		\xi(a):=[\ad \xi, a] .
	\end{equation*}
	It is sufficient to verify this fact for $a=a\left(\zeta_1, \zeta_2, \zeta_3\right)$:
	\begin{equation*}
		\begin{split}
			{[\ad a(\xi), b] } & =[\ad[\zeta_1[\zeta_2[\zeta_3, \xi]]], b] \quad (\text{since } \ad[\xi, \eta]=[\ad \xi, \ad \eta]) \\
			& =[[\ad \zeta_1[\ad \zeta_2[\ad \zeta_3, \ad \xi]]], b] \\
			& =[\zeta_1 \zeta_2 \zeta_3 \xi, b]-[\zeta_1 \zeta_2 \xi \zeta_3, b]-[\zeta_1 \xi \zeta_2 \zeta_3, b]+[\xi \zeta_1 \zeta_2 \zeta_3, b] \\
			& =a[\ad \xi, b]+[\zeta_1[\zeta_2 \xi] \zeta_3, b] .
		\end{split}
	\end{equation*}
	But
	\begin{equation*}
		\begin{aligned}
			\xi(a) & =[\ad \xi, a]=\xi \zeta_1 \zeta_2 \zeta_3-\zeta_1 \zeta_2 \zeta_3 \xi-T_4\left(\xi, \zeta_1, \zeta_2, \zeta_3\right) \\
			& =\zeta_1 \xi \zeta_2 \zeta_3-\zeta_1 \zeta_2 \xi \zeta_3=\zeta_1[\xi, \zeta_2] \xi_3 \in A(\mathfrak{G}),
		\end{aligned}
	\end{equation*}
	which consequently commutes with $b \in A(\mathfrak{G}) \Rightarrow [\ad a(\xi), b] = a [\ad \xi, b]$.
	\cref{thm:3} is proved.
\end{proof}


\begin{proof}[Proof of \cref{thm:4}]
	For this proof, we will use only the equality
	\begin{equation*}
		\xi(a) = \zeta_1 [\xi \zeta_2] \zeta_3
	\end{equation*}
	for $\xi \in \mathfrak{G}$, $a = a(\zeta_1,\zeta_2,\zeta_3) \in A(\mathfrak{G})$.
	Then
	\begin{equation*}
		\xi(a) \eta = \zeta_1 [\xi \zeta_2] \zeta_3(\eta)
	\end{equation*}
	by Jacobi's identity, and also
	\begin{equation*}
		\begin{split}
			& = \zeta_1 [[\xi \zeta_2] \zeta_3](\eta) + \zeta_1 \zeta_3 [\xi \zeta_2](\eta) \\
			& = - \zeta_1 \eta \zeta_3 \zeta_2 (\xi) + \zeta_1 \zeta_3 \eta \zeta_2 (\xi) \\
			& = (\zeta_1 \eta \zeta_2 \zeta_3 - \zeta_1 \zeta_2 \eta \zeta_3) (\xi)
			= \zeta_1 [\eta \zeta_2] \zeta_3 (\xi) 
			\\
			& = \eta(a) \xi. \qedhere
		\end{split}
	\end{equation*}
\end{proof}



\begin{thebibliography}{99}
	
	\bibitem{1}
	I. M. Gel'fand and A. A. Kirillov, 
	\textit{Sur les corps li\'es aux alg\`ebres enveloppantes des alg\`ebres de Lie}, 
	Publ. Math. IHES, No. 31 (1966), 509--523.
	
	\bibitem{2}
	A. A. Kirillov, 
	\textit{Elements of representation theory} (in Russian), 2nd ed., Nauka, Moscow, 1978.
	
	\bibitem{3}
	A. A. Kirillov, 
	\textit{Infinite-dimensional groups, their representations, orbits, invariants}, 
	Proc. Intern. Congress Math., Helsinki, 1978, pp. 705--708.
	
	\bibitem{4}
	A. A. Kirillov, 
	\textit{On identities in the Lie algebra of Hamiltonian vector fields on the plane}, 
	Keldysh Inst. Prilkl. Mat., preprint No. 121, 1983.
	
	\bibitem{5}
	A. A. Kirillov, M. L. Kontsevich, and A. I. Molev, 
	\textit{Algebras of intermediate growth}, 
	Sel. Math. Sov. 9:2 (1990), 137--153. (Originally published in Keldysh Inst. Prilkl. Mat. preprint No. 39 (1983).)
	
	\bibitem{6}
	A. I. Shirshov, 
	\textit{On subalgebras of Lie algebras} (in Russian), 
	Mat. Sb. 33(75) (1953), 441--452.
	
\end{thebibliography}

	
	
\end{document}
