\documentclass{article}

\usepackage[utf8]{inputenc}
\usepackage[width=14cm]{geometry}
\usepackage{amsmath,amsfonts,amssymb,amsthm}
\usepackage[hidelinks,pdfusetitle]{hyperref}
\usepackage[capitalise]{cleveref}
\usepackage{mathrsfs}
\usepackage{enumerate}

\newtheorem*{theorem*}{Theorem}
\newtheorem{theorem}{Theorem}
\newtheorem{corollary}{Corollary}[section]
\newtheorem{lemma}{Lemma}[section]
\newtheorem{definition}{Definition}
\newtheorem{remark}{Remark}[section]
\newtheorem{example}{Example}[section]

\title{On the local controllability of a \\ scalar-input control system}
\author{Gianna Stefani}
\date{
	Dipartimento di Sistemi e Informatica\\
	Università di Firenze\\
	Firenze, Italia
}

\newcommand{\ad}{\operatorname{ad}}
\newcommand{\interior}{\operatorname{int}}
\newcommand{\R}{\mathbb{R}}
\newcommand{\N}{\mathbb{N}}
\newcommand{\dd}{\mathrm{d}}

\begin{document}
	
\noindent
Theory and Applications of Nonlinear Control Systems \\
C.I.\ Byrnes and A.\ Lindquist (editors) \\
\copyright Elsevier Science Publishers B.V.\ (North-Holland), 1986

{\let\newpage\relax\maketitle}

\maketitle

\renewcommand*\abstractname{}

\begin{abstract}
	A necessary condition of local controllability of a scalar-input control system is proven. 
	The proof is based on a general result also proven in this work on the construction of a ``suitable chart''.
	
	The necessary condition of local controllability is extended to a necessary condition of local controllability along a reference trajectory.
\end{abstract}

\section{Introduction}

Let us consider the following control system
\begin{equation} \label{eq:1}
	\begin{cases}
		\dot{x}(t) = f_0(x(t)) + u(t) f_1(x(t)) \\
		x(0) = x_0
	\end{cases}
\end{equation}
where the state $x(t)$ belongs to a $C^\omega$ manifold $M$, $f_0$ and $f_1$ are $C^\omega$ vector fields such that $f_0(x_0) = 0$ and $f_1(x_0) \neq 0$.
The control map $t \to u(t)$ belongs to the collection $U$ of the integrable maps defined on $[0,1]$ with values in $[-1,1]$.

The system \eqref{eq:1} is said to be \emph{locally controllable} if, for each $t>0$, $x_0$ belongs to $\interior R(x_0,t)$ (= the interior of the reachable set from $x_0$ at time $t$). To the author's knowledge the necessary conditions of local controllability are the following ones (see \cite{2,3})
\begin{equation} \label{eq:2}
	L(x_0) = T_{x_0} M
\end{equation}
($L$ is the Lie algebra generated by $f_0$ and $f_1$)
\begin{equation} \label{eq:3}
	\ad^2_{f_1} f_0 (x_0) \in S^1(x_0)
\end{equation}
$\ad^k_{f_1} f_0$ is defined by $\ad_{f_1} f_0 = [f_1,f_0]$, $\ad^k_{f_1} f_0 = [f_1, \ad^{k-1}_{f_1} f_0]$, and $S^k$ is the subspace of $L$ consisting of all the brackets containing $f_1$ at most $k$ times.

In this paper the following generalization of \eqref{eq:3} will be proven.

\begin{theorem} \label{thm:1}
	 A necessary condition for the local controllability of the system \eqref{eq:1} is
	\begin{equation} \label{eq:4}
		\forall r \text{ even,} \quad
		\operatorname{ad}_{f_1}^r f_0\left(x_0\right) \in S^{r-1}\left(x_0\right).
	\end{equation}
\end{theorem}

The proof of the theorem is based on a general result also proven in the paper on the construction of a ``suitable chart''.

To be more precise, let $F=\left\{F_i\right\}_{i \geq 0}$ be an increasing filtration at $x_0$ of $L$, that is a sequence of linear subspaces of $L$ such that
\begin{enumerate}[i)]
	\item $F_i \subseteq F_{i+1}$
	\item $F_{\infty} \equiv \underset{i \geq 0}{\cup} F_i=L$
	\item $\left[F_i, F_j\right] \subseteq F_{i+j}$
	\item $\forall f \in F_0, \quad f\left(x_0\right)=0$.
\end{enumerate}
For each $f \in L$ the weight (with respect to $F$) is defined by $w(f)=\min (i: f \in F_i)$. 
Obviously $w([f, g]) \leq w(f)+w(g)$, $\forall f, g \in L$.

Identifying a vector field $f$ with the derivation $L_f$, the associative, non commutative algebra of differential operators $A$, generated on $\R$ by $L$, acting on the $C^\omega$ functions defined on $M$, can be considered.
The filtration $F$ induces a filtration $A=\{A_i\}_{i \geq 0}$ of the algebra $A$ by $A_i=\operatorname{span}\{D=L_{f_1} \circ \ldots \circ L_{f_j}: \sum_{k=1}^j w\left(f_k\right) \leq i\}$. 
The definition of weight can be extended to each $D \in A$ in the obvious way $w(D)=\min \left\{i: D \in A_i\right\}$.

\begin{theorem} \label{thm:2}
	Let $\varphi: M \to \R$ be a $C^\omega$ function such that $L_f \varphi\left(x_0\right)=0$ for each $f \in F_r$. 
	There is a $C^\omega$ function $\tilde{\varphi}$ such that $j_1 \tilde{\varphi}\left(x_0\right)=j_1 \varphi\left(x_0\right)$ (i.e. $\varphi$ and $\tilde{\varphi}$ coincide up to the first order) and $D \tilde{\varphi}\left(x_0\right)=0$ for each $D \in A_r$.
\end{theorem} 

\begin{remark}
	If $r=+\infty$ so that $L\left(x_0\right) \neq T_{x_0} M$, the construction of $\tilde{\varphi}$ provides an iterative method to construct the \emph{local} integral manifold of $L$ through $x_0$ (see \cref{rk:2.1}).
\end{remark} 

Finally if $f_0\left(x_0\right) \neq 0$, \eqref{eq:4} becomes a necessary condition of local controllability along a reference trajectory in the following sense

\begin{theorem} \label{thm:3}
	Let $\left(t, x_0\right) \rightarrow \exp  tf_0\left(x_0\right)$ be the local flow of $f_0$. 
	If \eqref{eq:4} doesn't hold, then there is $T>0$ such that $\forall t \leq T$, $\exp tf_0\left(x_0\right) \notin \interior R\left(x_0, t\right)$.
\end{theorem} 

\section{Proof of \texorpdfstring{\cref{thm:2}}{Theorem 2}}
\label{sec:2}

Let $x=\left\{x_1, \ldots, x_n\right\}$ be a chart at $x_0$ adapted to $F$ in the following sense
\begin{enumerate}[i)]
	\item $x\left(x_0\right)=0 \in \mathbb{R}^n$
	\item $\operatorname{span}\left\{\frac{\partial}{\partial x_1}\left(x_0\right), \ldots, \frac{\partial}{\partial x_{m_i}}\left(x_0\right)\right\}=F_i\left(x_0\right)$
\end{enumerate}
$m_i$ being the dimension of $F_i\left(x_0\right)$.

Starting from any chart at $x_0$, an adapted chart can be obtained by a linear change of coordinates.

The monomial $x_1^{\nu_1} x_2^{\nu_2} \ldots x_n^{\nu_n}$, $\nu_i \in \N$, will be denoted by $x^\nu$. 
For each $r=1, \ldots, n$, the weight $w_r$ is defined as $w_r=\min \left\{i: \frac{\partial}{\partial x_r}\left(x_0\right) \in F_i\left(x_0\right)\right\}$ ($w_r=+\infty$ if $\frac{\partial}{\partial x_r}\left(x_0\right) \notin L\left(x_0\right)$) and the monomial $x^\nu$ is said to have degree $|\nu|=\sum_{i=1}^n \nu_i$ and weight $\|\nu\|=\sum_{i=1}^n \nu_i w_i$.

Let $m=\operatorname{dim} L\left(x_0\right)$ ($m=n$ if $L$ has full rank at $x_0$). 
Let $g_1, \ldots, g_m \in L$ be such that $g_i\left(x_0\right)=\frac{\partial}{\partial x^i}\left(x_0\right)$, $i=1, \ldots, m$. 
The differential operator $L_{g_m}^{\nu_m} \circ \dotsc \circ L_{g_1}^{\nu_1} \in A$ is denoted by $D^\nu$ and $|\nu|$ ($\|\nu\|$) is the degree (the weight) of $D^\nu$.

Moreover the set of all multiindex of ``degree $s$ and weight $r$'' is denoted by $\chi(s, r)$ i.e.
\begin{equation*}
	\chi(s, r)=\left\{\left(\nu_1, \ldots, \nu_m\right): \nu_i \geq 0, |\nu|=s, \|\nu\| \leq r\right\} .
\end{equation*}
Notice that $\chi(s, r)=\emptyset$ if $s>r$.

Let $\varphi$ be as in the \cref{thm:2}, that is $L_f \varphi\left(x_0\right)=0, \forall f \in F_r$. 
Without loss of generality we can suppose $\varphi\left(x_0\right)=0$.

Let us define recursively for $s \leq r$ :
\begin{equation*}
	\varphi_1=\varphi, \quad 
	\varphi_s=\varphi_{s-1}-\sum_{\nu \in \chi(s, r)} \frac{1}{\nu!} D^\nu \varphi_{s-1}\left(x_0\right) x^\nu,
\end{equation*}
where $\nu !=\nu_{1} ! \ldots \nu_{n} !$.
It is clear that $\varphi$ and $\varphi_s$ coincide up to the first order.

\bigskip

\textbf{a)} \underline{$r < +\infty$}

Let us prove by induction on $s \leq r$ that
\begin{equation} \label{eq:5}
	L_{f_1} \circ \ldots \circ L_{f_s} \varphi_s\left(x_0\right)=0 
	\quad \text { if } \quad 
	\sum_{j=1}^s w\left(f_j\right) \leq r
\end{equation}
For $s=1$, \eqref{eq:5} follows by the hypothesis.

Let $f_1, \ldots, f_{s+1}$ be such that $\sum_{j=1}^{s+1} w\left(f_j\right) \leq r$.
\begin{equation*}
	L_{f_1} \circ \cdots \circ L_{f_{s+1}} \varphi_{s+1}\left(x_0\right)
	= L_{f_2} \circ L_{f_1} \cdots \circ L_{f_{s+1}} \varphi_{s+1}\left(x_0\right)+L_{\left[f_1, f_2\right]} \circ \ldots \circ L_{f_{s+1}} \varphi_{s+1}\left(x_0\right) \text {. }
\end{equation*}
Being $\varphi_{s+1}=\varphi_s + (\text{polynomial of order } s+1)$, we get $L_{\left[f_1, f_2\right]} \cdot \ldots \circ L_{f_{s+1}} \varphi_{s+1}\left(x_0\right)=$ $=L_{\left[f_1, f_2\right]} \circ \cdots \circ L_{f_{s+1}} \varphi_s\left(x_0\right)=0$ by the inductive hypothesis, so that
$L_{f_1} \circ L_{f_2} \circ \cdots \circ L_{f_{s+1}} \varphi_{s+1}\left(x_0\right)=L_{f_2} \circ L_{f_1} \circ \cdots \circ L_{f_{s+1}} \varphi_{s+1}\left(x_0\right)$.
In an analogous way it is possible to show that $L_{f_1} \circ \cdots \circ L_{f_{s+1}} \varphi_{s+1}\left(x_0\right)=L_{f_{\sigma(1)}} \circ \cdots \circ L_{f_{\sigma(s+1)}} \varphi_{s+1}\left(x_0\right)$ for each permutation $\sigma$ of $[1, \ldots, s+1]$. 
This means that $L_{f_1} \circ \cdots \circ L_{f_{s+1}} \varphi_{s+1}\left(x_0\right)$ is a linear combination of terms of type $D^\nu \varphi_{s+1}\left(x_0\right)$ with $|\nu|=s+1$ and $\|\nu\| \leq r$.

If $\left|\nu'\right|=s+1$, $D^\nu x^{\nu'}\left(x_0\right)=\begin{cases}0 & \text { if } \nu' \neq \nu \\ \nu! & \text { if } \nu'=\nu\end{cases}$.

Hence $D^\nu \varphi_{s+1}\left(x_0\right)=D^\nu \varphi_s\left(x_0\right)-\sum_{\nu' \in \chi(s+1, r)} \frac{D^{\nu'} \varphi_s\left(x_0\right)}{\nu'!} D^\nu x^{\nu'}=0$.

Let us define $\tilde{\varphi}=\varphi_r$.

The previous arguments show that $D \tilde{\varphi}\left(x_0\right)=0$ if $D$ is a differential operator of degree and weight at most $r$. 
To get the proof it is sufficient to prove by induction on $i$ that $L_{f_1} \circ \cdots \circ L_{f_{r+i}} \tilde{\varphi}\left(x_0\right)=0$ if $\sum_{j=1}^{r+i} w\left(f_j\right) \leq r$. 
The last condition implies that there is $k \in \{ 1, \ldots, r+i \}$ such that $f_k \in F_0$. 
We get
$L_{f_1} \circ \cdots \circ L_{f_{r+i}} \tilde{\varphi}\left(x_0\right)=L_{f_k} \circ D_1 \tilde{\varphi}\left(x_0\right)+D_2 \tilde{\varphi}\left(x_0\right)$, where $D_1$ and $D_2$ are differential operator of degree $r+i-1$ and weight at most $r$. Hence $L_{f_k} \circ D_1 \tilde{\varphi}\left(x_0\right)=0$ by $f_k\left(x_0\right)=0$ and $D_2 \tilde{\varphi}\left(x_0\right)=0$ by the induction hypothesis and the theorem is proven.

\bigskip
\textbf{b)} \underline{$r=+\infty$}

Let $N$ be the \emph{local} integral manifold of $L$ through $x_0$. By a Nagano's theorem $N$ exists and its dimension is $m$.

Let $\left\{x_1, \ldots, x_n\right\}$ be adapted to $F$ and such that $x_{m+1}=\ldots=x_n=0$ are the equations of $N$.

Let the function $\psi$ be defined by $\psi\left(x_1, \ldots, x_n\right)=\varphi\left(x_1, \ldots, x_m, 0 \ldots 0\right)$.
As $\psi_{\mid N}=\varphi_{\mid N}$ and each $f \in L$ is tangent to $N$ it is sufficient to prove that the sequence $\left\{\varphi_s\right\}_{s \geq 1}$ converges to $\tilde{\varphi}=\varphi-\psi$, in fact $L_f \varphi_{\mid N}=L_f \psi_{\mid N}, \forall f \in L$. 
Let $\Psi_s$ be the Taylor expansion of $\psi$ up to order s. 
By $L_f \varphi\left(x_0\right)=0 \forall f \in L$,
it follows that $\psi_1=0$ and hence $\varphi_1=\varphi-\psi_1$. 
By the inductive hypothesis, let $\varphi_s=\varphi-\psi_s$. We get $\varphi_{s+1}=\varphi-\psi_s-\sum_{|\nu|=s+1} D^\nu\left(\varphi-\psi_s\right)\left(x_0\right) / \nu ! x^\nu$. 
But $D^\nu \varphi\left(x_0\right)=D^\nu \psi\left(x_0\right)$, so that $D^\nu\left(\varphi-\psi_s\right)\left(x_0\right)=D^\nu\left(\psi-\psi_s\right)\left(x_0\right)=\frac{\partial^{s+1}}{\partial x^\nu} \psi\left(x_0\right)$ and $\psi_s+\sum_{|\nu|=s+1} D^\nu\left(\varphi-\psi_s\right)\left(x_0\right) / \nu ! x^\nu=\psi_{s+1}$.

\begin{remark}
	\label{rk:2.1}
	The previous result gives an iterative method to construct the equations of $N$ in the following sense.
	
	Starting from any chart at $x_0$ it is possible, by a linear change of coordinates, to get a chart $\left\{x_1, \ldots, x_n\right\}$ adapted to $F$ and such that $L_f x_i\left(x_0\right)=0$, $i=m+1, \ldots, n$. 
	Applying the theorem and denoting $x_i-\tilde{x}_i\left(x_1, \ldots, x_m\right)$ by $\psi_i$, we get that $x_i=\psi_i\left(x_1, \ldots, x_m\right), i=m+1, \ldots, n$, are parametric equations of $N$, where the parameters are ``the coordinates of the tangent space $T_{x_0} N$''.
\end{remark}

\begin{example}
	Let $M=\R^2$, $f_0(x, y)=(x, 2 y)$, $f_1(x, y)=(1, x+y)$. 
	Let the filtration $F$ be defined by: $F_0 = \operatorname{span} \left\{f_0\right\}$, $F_i=F_0+S^i$.
	We get $\left[f_1, f_0\right](x, y)=(1, x)$, $\operatorname{ad}^k_{f_0}\left(f_1\right)(x, y)=(-1)^{k}(1, x)$ and $\left[f_1,\left[f_0, f_1\right]\right](x, y)=(0, x)$, so that $F_1(0,0)=F_2(0,0)=\operatorname{span}\{(1,0)\}$.
	Applying the \cref{thm:2} for $r=2$, we get: $\tilde{y}=y-\frac{1}{2} f_1^2 y(0,0) x^2=y-\frac{1}{2} x^2$ and $L_{f_1} \circ L_{f_0}^h \circ L_{f_1} \circ L_{f_0}^k \tilde{y}(0,0)=0 \quad \forall h, k \geq 0$.
\end{example}

\begin{example}
	Let $M=\left\{(x, y, z) \in \mathbb{R}^3:|x|,|y|,|z|<1\right\}$, $f_0(x, y, z)=\left(0, x, x^2 /(1+x y)\right))$, $f_1(x, y, z)=(1, y,(y+x y) /(1+x y))$.
	
	Let the filtration $F$ be defined by $F_0=\{0\}$, $F_1=\operatorname{span}\left\{f_0, f_1\right\}$, $F_2=F_1+\operatorname{span}\left\{\left[f_0, f_1\right]\right\}$, $F_3=L$. 
	We get $L(0)=F_2(0)=\left\{(\lambda, \mu, 0):(\lambda, \mu) \in \mathbb{R}^2\right\}$.
	It is easy to see that $L_{f_0}(z-\log (1+x y))=L_{f_1}(z-\log (1+x y))=0$. 
	In this case $\tilde{z}=z-\log (1+x y)$ and $z-z_s=\sum_{i=1}^s(-1)^{i-1} \frac{(x y)^i}{i}$ is the Taylor expansion of $\log (1+x y)$ up to order $s$.
\end{example}

\section{Proof of \texorpdfstring{\cref{thm:1}}{Theorem 1}}
\label{sec:3}

Let $r>0$ be even and let $\operatorname{ad}_{f_1}^r f_0\left(x_0\right) \notin S^{r-1}\left(x_0\right)$. 
There is a $C^\omega$ function $\varphi$ such that $\varphi\left(x_0\right)=0$, $L_{\operatorname{ad}_{f_1}^r f_0} \varphi\left(x_0\right)=1$ and $L_f \varphi\left(x_0\right)=0, \forall f \in S^{r-1}$. 
By the \cref{thm:2} applied to the filtration $F_0=\operatorname{span}\left(f_0\right)$, $F_{i}=F_0+S^i$, there is $\tilde{\varphi}$ with the same properties and $D \tilde{\varphi}\left(x_0\right)=0$ for each $D \in A_{r-1}$. 
In what follows we set $\varphi=\tilde{\varphi}$.
Moreover let $\psi$ be a $C^\omega$ function such that $L_{f_1} \psi\left(x_0\right)=1$.

We shall prove that there is $T>0$ with the following property:
\begin{equation}
	\tag{P}
	\label{eq:P}
	\text{If $x \in R\left(x_0, t\right)$, $t \leq T$, is such that $\psi(x)=0$, then $\varphi(x) \geq 0$.}
\end{equation}
This shows that $x_0 \notin \interior R\left(x_0, t\right)$. 
To get the proof we need some technical results on integral inequalities.

Let the integral operator $I: U\to U$ be defined by $I u(t)=\int_0^t u(s) \dd s$, $\forall u \in U$.
Moreover for each $u \in U$ we define $I_u : U \to U$ by $I_u w = I(u w)$, $\forall w \in U$.
Let $v=I u$. It is not difficult to see that
\begin{align}
	\label{eq:6}
	I^{k+1} w(t) & = \int_0^t \frac{(t-s)^k}{k !} w(s) \dd s \quad \forall w \in U \\
	\label{eq:7}
	I_u^k I w(t) & = \int_0^t \frac{(v(t)-v(s))^k}{k !} w(s) \dd s \quad \forall w \in U
\end{align}
Let $h=\left(h_1, \ldots, h_j\right)$ be a multiindex with $h_1, \ldots, h_j \geq 0$, we denote $|h|=h_1+\ldots+h_j$. Moreover, in what follows we denote by $\mathfrak{i}: [0,1] \to [0,1]$ the identity map $\mathfrak{i}(t) \equiv 1$ and for sake of simplicity we define
\begin{equation}
	\label{eq:8}
	\left(I_u^{k_j} \circ I^{h_j} \circ \cdots \circ I_u^{k_1} \circ I^{h_1}\right) \mathfrak{i}(t)=p(k, h, u, t)
\end{equation}
It is not difficult to see that $\forall u \in U$,
\begin{equation}
	\label{eq:9}
	|p(k, h, u, t)| \leq \frac{2^{\ell}}{\ell !} \ell^2 t^{\ell-2} \int_0^t|v(s)| \dd s
\end{equation}
if $|k| \geq 1$, $|k|+|h|=\ell \geq h_1+2$, $v=I u$.

\begin{lemma}
	Let $j \geq 1$; $h_1, k_j \geq 0$; $k_1, \ldots, k_{j-1}, h_2, \ldots, h_j \geq 1$; $|k|=r \geq 1$; $|k|+|h|=\ell \geq r+1$; $v=I u$. Then
	\begin{equation}
		\label{eq:10}
		|p(k, h, u, t)| \leq \frac{r^r (2 r)! (r+1)^{\ell+1}}{\ell !} \sum_{i=0}^r|v(t)|^{r-i} t^{\ell-r-i / r}\left(\int_0^t|v(s)|^r \dd s\right)^{i / r} .
	\end{equation}
\end{lemma}

\begin{proof}
	Let $h_1 \geq 1$. 
	Then 
	\begin{equation*}
		\begin{split}
			|p &(k, h, u, t)| = |(I_u^{k_j} \circ I^{h_j} \ldots \ldots I_u^{k_1} \circ I^{h_1}) \mathfrak{i}(t)| \\
			& \leq \int_0^t \frac{|v(t)-v(s)|^{k_j}}{k_{j} !} |(I^{h_j-1} \circ \ldots \circ I_u^{k_1} \circ I^{h_1}) \mathfrak{i}(s)| \dd s \\
			& \leq \left(\int_0^t \frac{(|v(t)|+|v(s)|)^{k_j}}{k_j!} \dd s \right) \frac{t^{h_j-1}}{(h_j-1)!} \left(\int_0^t \frac{(|v(t)|+|v(s)|)^{k_{j-1}}}{k_{j-1}!} \dd s \right) \dotsb \\
			& \quad \quad \dotsb \frac{t^{h_2-1}}{(h_2-1)!} \left(\int_0^t \frac{(|v(t)|+|v(s)|)^{k_1}}{k_1!} \dd s \right)\frac{t^{h_1-1}}{(h_1-1)!} \\
			& \leq \frac{t^{\ell-j-r}}{(h_1-1)!\dotsb(h_j-1)!} \sum_{i_1=0}^{k_1} \dotsb \sum_{i_j=0}^{k_j} \binom{k_1}{i_1} \dotsb \binom{k_j}{i_j} \frac{|v(t)|^{r-i_1-\dotsb-i_j}}{k_1!\dotsb k_j!} \left(\int_0^t |v(s)|^{i_1} \dd s\right) \dotsb \\
			& \quad \quad \dotsb \left(\int_0^t |v(s)|^{i_j} \dd s\right).
		\end{split}
	\end{equation*}
	Using the Hölder inequality
	\begin{equation}
		\label{eq:11}
		\int_0^t |v(s)|^i \dd s \leq t^{1-i/r} \left( \int_0^t |v(s)|^r \dd s\right)^{i/r},\quad i \leq r
	\end{equation}
	we get
	\begin{equation*}
		\begin{split}
			|p&(k,h,u,t)| \leq \frac{t^{\ell-j-r}}{(h_1-1)! \dotsb (h_j-1)!} \sum_{i_1=0}^{k_1} \dotsb \\
			& \quad \dotsb \sum_{i_j=0}^{k_j} \frac{|v(t)|^{r-i_1-\dotsb-i_j}}{i_1!\dotsb i_j!} t^{1-i_1/r}
			\left(\int_0^t |v(s)|^r \dd s\right)^{i_1/r} \dotsb t^{1-i_j/r} \left(\int_0^t |v(s)|^r \dd s\right)^{i_j/r}
		\end{split}
	\end{equation*}
	Using the equality
	\begin{equation}
		\label{eq:12}
		\sum_{\substack{i_1,\dotsb,i_j\geq0\\i_1+\dotsb+i_j=i}} \frac{i!}{i_1! \dotsb i_j!} = j^i
		\quad \forall j, i \geq 1
	\end{equation}
	we get
	\begin{equation*}
		\begin{split}
			|p(k,h,u,t)| 
			& \leq \sum_{i=0}^r \frac{j^i}{i!} |v(t)|^{r-i} \frac{t^{\ell-r-i/r}}{(h_1-1)! \dotsb (h_j-1)!} \left(\int_0^t |v(s)|^r \dd s\right)^{i/r} \\
			& \leq \frac{r^r (2r)! (r+1)^{\ell+1}}{\ell!} \sum_{i=0}^r |v(t)|^{r-i} t^{\ell-r-i/r} \left(\int_0^t |v(s)|^r \dd s\right)^{i/r}.
		\end{split}
	\end{equation*}
	If $h_1 = 0$, in an analogous way we obtain
	\begin{equation*}
		\begin{split}
			|&p(k,h,u,t)| 
			\leq \sum_{i=0}^{r-k_1} \frac{(j-1)^i}{i!} \frac{|v(t)|^{r-k_1-i}}{k_1!} \frac{t^{\ell-r-(k_1+i)/r}}{(h_2-1)! \dotsb (h_j-1)!} \left(\int_0^t |v(s)|^r \dd s\right)^{(i+k_1)/r} \\
			& = \sum_{i=k_1}^{r} \frac{(j-1)^{i-k_1}}{(i-k_1)!} \frac{|v(t)|^{r-i}}{k_1!} \frac{t^{\ell-r-i/r}}{(h_2-1)! \dotsb (h_j-1)!} \left(\int_0^t |v(s)|^r \dd s\right)^{i/r} \\
			& \leq(r-1)^{r-k_1}\left(2 r-k_1-1\right) ! \frac{(r+1)^{\ell+1}}{\ell !} \sum_{i=k_1}^r|v(t)|^{r-i} t^{\ell-r-i / r}\left(\int_0^t|v(s)|^r \dd s\right)^{i / r}
		\end{split}
	\end{equation*}
	Hence \eqref{eq:10} is true also in this case.
\end{proof}

\begin{corollary}
	Let $j \geq 1$; $h_1, k_j \geq 0$; $k_1, \ldots, k_{j-1}, h_2, \ldots, h_j \geq 1$; $|h|+|k|=\ell$.
	
	If $|k|>r \geq 1$ or $|k|=r \geq 1$ and $k_j=0$, then
	\begin{equation}
		\label{eq:13}
		|p(k, h, u, t)| \leq (2 r) ! r^{r+1} \frac{(2 r+2)^\ell}{(\ell-1) !} t^{\ell-r-1} \int_0^t|v(s)|^r \dd s
	\end{equation}
\end{corollary}

\begin{proof}
	There is $j' \geq 1$, $k_{j'}' \leq k_{j'}$ and $k'=\left(k_1, \ldots, k_{j'-1}, k_{j'}'\right)$ such that $\left|k'\right|=r$. 
	Hence setting $h'=\left(h_1, \ldots, h_{j'}\right)$ we get $\ell-r-|h'| \geq 1$ and
	\begin{equation*}
		\begin{aligned}
			& |p(k, h, u, t)| \leq \frac{t^{\ell-r-|h'|-1}}{\left(l-r-|h'|-1\right) !} \int_0^t\left|p\left(k', h', u, s\right)\right| \dd s \\
			& \leq \frac{t^{\ell-r-|h'|-1}}{\left(\ell-r-|h'|-1\right) !} \frac{r^r (2 r)! (r+1)^{r+|h'|+1}}{\left(r+|h'|\right) !} \sum_{i=0}^r t^{|h'|-i / r}\left(\int_0^t|v(s)|^r \dd s\right)^{i / r} \int_0^t|v(s)|^{r-i} \dd s \\
			& \leq \text {(by \eqref{eq:11} and \eqref{eq:12}) } \frac{2^{\ell-1}}{(l-1) !} r^r (2 r) !(r+1)^{r+|h'|+1} t^{\ell-r-1} \sum_{i=0}^r t^{\frac{i}{r} - \frac{i}{r}}\left(\int_0^t|v(s)|^r \dd s\right)^{\frac{i}{r} + \frac{r-i}{r}}  \\
			& \leq (2 r) ! r^{r+1} \frac{(2 r+2)^l}{(l-1) !} t^{\ell-r-1} \int_0^t|v(s)|^r \dd s.
		\end{aligned}
		\qedhere
	\end{equation*}
\end{proof}

\begin{corollary}
	Let $u$ be such that $|v(t)| \leq A \int_0^t|v(s)| \dd s$ for some $A \geq 1$ and let $j \geq 1$; $h_1 \geq 0$; $k_1, \ldots, k_j, h_2, \ldots, h_j \geq 1$; $|h|+|k|=\ell \geq r+1$; $|k|=r$. 
	Then
	\begin{equation}
		\label{eq:14}
		|p(k, h, u, t)| \leq (2 r) ! r^{r+1} A^r \frac{(r+1)^{\ell+1} t^{\ell-r-1}}{\ell !} \int_0^t|v(s)|^r \dd s
	\end{equation}
\end{corollary}

\begin{proof}
	\begin{equation*}
		\begin{split}
			& |p(k, h, u, t)| \leq \frac{r^r (2 r) !(r+1)^{\ell+1}}{\ell !} \sum_{i=0}^r A^{r-i}\left(\int_0^t|v(s)| \dd s\right)^{r-i} t^{\ell-r-i / r}\left(\int_0^t|v(s)|^r \dd s\right)^{i / r} \\
			& \leq (\text{by \eqref{eq:11}}) \frac{r^r (2 r) !(r+1)^{\ell+1}}{\ell !} A^r \sum_{i=0}^r t^{(1-\frac 1 r)(r-i)}\left(\int_0^t|v(s)|^r \dd s\right)^{1-\frac i r} t^{\ell-r-\frac i r}\left(\int_0^t|v(s)|^r \dd s\right)^{\frac i r} \\
			& \leq (2 r) ! r^{r+1} A^r \frac{(r+1)^{\ell+1} t^{\ell-r-1}}{\ell !} \int_0^t|v(s)|^r \dd s,
		\end{split}
	\end{equation*}
	using $t^{r-i} \leq 1$ since $t \leq 1$.
\end{proof}

	
Now let us prove the property \eqref{eq:P}.
Let $x(u, t)$ be the solution at time $t$ relative to the control map $u$. It is known \cite{1,3} that there is $T^*>0$ such that $\forall t \leq T^*$
\begin{equation} \label{eq:15}
	\varphi(x(u, t))=\sum_{|h|+|k| \geq 0}(D(k, h) \varphi)\left(x_0\right) p(k, h, u, t)
\end{equation}
where the differential operator $D(k, h)$ is defined by
\begin{equation} \label{eq:16}
	D(k, h)=L_{f_0}^{h_1} \circ L_{f_1}^{k_1} \cdots \circ L_{f_0}^{h_j} \circ L_{f_1}^{k_j}
\end{equation}
and $p(k, h, u, t)$ is defined by \eqref{eq:8}.

By the properties of $\varphi$, taking into account that $f_0\left(x_0\right)=0$ and
\begin{equation} \label{eq:17}
	\left(I_u^{r-i} \circ I \circ I_u^i\right) \mathfrak{i}(t) = \sum_{j=0}^{r-i} \binom{r-i}{j} v^j(t) \int_0^t(-1)^{r-i-j} \frac{v^{r-j}}{(r-i) ! i !}(\tau) \dd \tau
\end{equation}
and
\begin{equation} \label{eq:18}
	\operatorname{ad}_{f_1}^r f_0=\sum_{i=0}^r(-1)^{r-i} \binom{r}{i} L_{f_1}^i \circ L_{f_0} \circ L_{f_1}^{r-i}
\end{equation}
we get
\begin{equation*}
	\begin{aligned}
		|\varphi(x(u, t))| & = \Big| L_{f_1}^r \varphi\left(x_0\right) v^r(t)+\sum_{i=0}^r(-1)^{r-i} L_{f_1}^i \circ L_{f_0} \circ L_{f_1}^{r-i} \varphi\left(x_0\right) \int_0 ^t \frac{v^r(\tau) \dd \tau}{i !(r-i) !} \\
		& +\sum_{i=0}^r L_{f_1}^i \circ L_{f_0} \circ L_{f_1}^{r-i} \varphi\left(x_0\right) \sum_{j=1}^{r-i}(-1)^{r-i-j} \binom{r-i}{j} \frac{v^j(t)}{(r-i) ! i !}\int_0 ^t v^{r-j}(\tau) \dd \tau \\
		& +L_{f_1}^{r+1} \varphi\left(x_0\right) v^{r+1}(t)+\sum_{\substack{|k| \geq r\\ |k|+|h|\geq r+2}} D(k, h) \varphi\left(x_0\right) p(k, h, u, t) \Big| \\
		& \geq \text{(being $r$ even)} \quad \frac{1}{r !} \int_0^t v^r(\tau) \dd \tau -|\text{all other terms}|
	\end{aligned}
\end{equation*}
Being $\varphi, f_0, f_1$ analytic there is $K_{\varphi} \geq 1$ such that
\begin{equation}
	\label{eq:19}
	\left|D(k, h) \varphi\left(x_0\right)\right| \leq K_{\varphi}^{|k|+|h|}(|k|+|h|) !
\end{equation}
(see \cite{1,3}).
Hence if $t$ and $u$ are such that
\begin{equation}
	\label{eq:20}
	|v(t)| \leq A \int_0^t|v(\tau)| \dd \tau \text { for some } A \geq 1
\end{equation}
we get by \eqref{eq:13} and \eqref{eq:14}
\begin{equation*}
	|\varphi(x(u, t))| \geq \int_0^t v^r(\tau) \dd \tau \left(\frac{1}{r !}-\eta(t)\right)
\end{equation*}
where
\begin{equation*}
	\begin{split}
		\eta(t) & \leq \left(A K_{\varphi}\right)^r r ! t^{r-1}+\left(A K_{\varphi}\right)^{r+1}(r+1) ! t^r+\sum_{i=0}^r \sum_{j=1}^{r-i} \frac{t^j}{j ! i !}\left(A K_{\varphi}\right)^{r+1}(r+1) ! \\
		& \quad +\sum_{\substack{|k| \geq r \\
				\ell=|k|+|h| \geq r+2}} (2 r) ! r^{r+1} A^r \frac{(2 r+2)^{\ell+1}}{(\ell-1) !} t^{\ell-r-1} K_{\varphi}^{\ell} \ell !  \\
		& \leq (2 r) !\left(A K_{\varphi} r\right)^{r+1} t\left(t^{r-2}+t^{r-1}+\sum_{i=0}^r \sum_{j=1}^{r-i} \frac{t^{j-1}}{j ! i !}+\sum_{\ell \geq r+2} 2^\ell (2r+2)^{\ell+1} K_\varphi^\ell \ell t^{\ell-r-2}\right).
	\end{split}
\end{equation*}
The series $\sum_{\ell \geq r+2}\left(2(2 r+2) K_{\varphi}\right)^{\ell+1} \ell t^{\ell-r-2}$ converges to a $C^\omega$ function for $|t|<\frac{1}{2(2 r+2) K_{\varphi}}$, so that there is $T>0$ such that for each $t<T$ and each $u$ satisfying \eqref{eq:20}, $\eta(t)<\frac{1}{r !}$.
Hence we get the proof of the property \eqref{eq:P} by the following



\begin{lemma} \label{lem:3.2}
	Let $\psi$ be an analytic function such that $L_{f_1} \psi\left(x_0\right)=1$. 
	There is $T^{\star}>0$ and $A \geq 1$ such that if $t<T^*$ and $u$ satisfies $\psi(x(u, t))=0$, then $|v(t)| \leq A \int_0^t|v(\tau)| \dd \tau$.
\end{lemma}

\begin{proof}
	By \eqref{eq:15} and the fact that $f_0\left(x_0\right)=0$ we get
	\begin{equation*}
		\begin{split}
			\psi(x(u, t)) 
			& = L_{f_1} \psi\left(x_0\right) v(t)+\sum_{i \geq 1} L_{f_1} \circ L_{f_0}^i \psi\left(x_0\right)(I^i \circ I_u) \mathfrak{i} (t) \\
			& + \sum_{|k| \geq 2} D(k, h) \psi\left(x_0\right) p(k, h, u, t)=v(t)+\rho(t) .
		\end{split}
	\end{equation*}
	Hence if $\psi(x(u, t))=0$, by \eqref{eq:13} and \eqref{eq:19} we get
	\begin{equation*}
		\begin{aligned}
			|v(t)| & =|\rho(t)| \leq \sum_{\ell=|k|+|h| \geq 2} 2^{\ell} 2 	\frac{4^{\ell}}{(\ell-1) !} K_\psi^{\ell} \ell ! t^{\ell-2} \int_0^t|v(\tau)| \dd \tau \\
			& \leq \int_0^t|v(\tau)| \dd \tau \cdot \sum_{\ell \geq 2} 2(8 K_\psi)^{\ell} \ell t^{\ell-2} .
		\end{aligned}
	\end{equation*}
	The series $\sum_{\ell \geq 2}\left(8 K_\psi\right)^{\ell} \ell t^{\ell-2}$ converges to an increasing $C^\omega$ function for $0 \leq t <\frac{1}{8K_\psi}$ and the lemma is proven.
\end{proof}

\section{Proof of \texorpdfstring{\cref{thm:3}}{Theorem 3}}

The proof is a slight modification of the one of the \cref{thm:1}.
First of all we need a $C^\omega$ map $\tilde{\varphi}$ such that
\begin{enumerate}[i)]
	\item $L_{\operatorname{ad}_{f_1}^r f_0} \tilde{\varphi}\left(x_0\right)=1$
	\item \label{ii} $L_{f_0}^{h_1} \circ L_{f_1}^{k_1} \circ \cdots \circ L_{f_0}^{h_j} \circ L_{f_1}^{k_1} \tilde{\varphi} \left(x_0\right)=0$ if $|k| \leq r-1$.
\end{enumerate}
To get such a $\tilde{\varphi}$ let us start by a $\varphi$ such that
\begin{equation*}
	L_{\ad_{f_1}^r f_0} \varphi\left(x_0\right)=1, \quad L_f \varphi\left(x_0\right)=0 \quad \forall f \in S^{r-1}
\end{equation*}
and $L_{f_0} \varphi \equiv c$ (constant in a neighbourhood of $x_0$).

Let us apply \cref{thm:2} to the filtration $F$ given by $F_0=\{0\}$, $F_i=S^i$, starting from an adapted chart with the following properties:
\begin{enumerate}[a)]
	\item If $f_0\left(x_0\right)\notin S^{r-1}\left(x_0\right)$ we choose $x_1, \ldots, x_n$ such that $L_{f_0} x_i \equiv 0 \quad i=1, \ldots, m_{r-1}$
	\item If $f_0\left(x_0\right) \in S^{r-1}\left(x_0\right)$, let $k=\max \left\{j: f_0\left(x_0\right) \notin S^j\left(x_0\right)\right\}$; we choose $x_1, \ldots, x_n$ such that $L_{f_0} x_{m_k+1} \equiv 1$ and $L_{f_0} x_j \equiv 0 \quad j \neq m_k+1$.
\end{enumerate}
In both cases a) and b) we get
\begin{equation}
	\label{eq:21}
	L_{f_0} \tilde{\varphi}=L_{f_0} \varphi=c
\end{equation}
In the case a) $\tilde{\varphi}=\varphi+$ (a function depending only on $x_1, \ldots, x_{m_r}$) and \eqref{eq:21} can be easily proven.

Concerning the case b) we shall prove $L_{f_0} \varphi_s=L_{f_0} \varphi$ by induction on $s$.
\begin{equation*}
	L_{f_0} \varphi_1=L_{f_0} \varphi=c.
	\quad
	 L_{f_0} \varphi_{s+1}= \text{(see \cref{sec:2}) } 
	 L_{f_0} \varphi_s-\sum_{\nu \in \chi(s+1, r-1)} \frac{D^\nu \varphi_s\left(x_0\right)}{\nu!} L_{f_0} x^\nu.
\end{equation*}
If $\nu_k=0$, then $L_{f_0} x^\nu = 0$.
If $\nu_k \neq 0$, then 
\begin{equation*}
	\begin{split}
		D^\nu \varphi_s(x_0)
		& = L_{g_k} \circ L_{g_{m_r}}^{\nu_{m_r}} \circ \dotsb \circ L^{\nu_k-1}_{g_k} \circ \dotsb \circ L_{g_1}^{\nu_1} \varphi_s (x_0) \\
		& = \text{(by the choice of the chart) }
		L_{f_0} \circ D^{\nu'} \varphi_s(x_0) \\
		& = D^{\nu'} \circ f_0 \varphi_s(x_0) + D^{\nu''} \varphi_s(x_0)
	\end{split}
\end{equation*}
where $D^{\nu'}, D^{\nu''} \in A^{r-1}$.
Hence $D^{\nu'} \circ f_0 \varphi_s(x_0) = D^{\nu'} c = 0$, $D^{\nu''} \varphi_s(x_0) = 0$ and \eqref{eq:21} is proven.

By \eqref{eq:21} the proof of \ref{ii}) follows easily.

In what follows we set $\varphi=\tilde{\varphi}$.

Let $\psi$ be a $C^\omega$ function such that $L_{f_1} \psi\left(x_0\right)=1$.
We shall prove that there is $T>0$ with the following property:
\begin{equation}
	\label{eq:P'} \tag{P'}
	\begin{split}
		\text{If } & x \in R\left(x_0, t\right), t \leq T, \text{ is such that } \psi(x)=\psi\left(\exp t f_0\left(x_0\right)\right), \\
		& \text{then } \varphi(x) \geq \varphi(\exp t f_0(x_0)).
	\end{split}
\end{equation}
In fact $\varphi(x(u,t))-\varphi(\exp t f_0(x_0)) = \sum_{|k|\geq r} D(k,h) \varphi(x_0) p(k,h,u,t)$ and we get the property \eqref{eq:P'} as the property \eqref{eq:P} in \cref{sec:3} by the following

\begin{lemma}
	Let $\psi$ be an analytic function such that $L_{f_1} \psi\left(x_0\right)=1$. There is $T^*>0$ and $A \geq 1$ such that if $t<T^*$ and $u$ satisfies $\psi(x(u, t))=\psi(\exp t f_0(x_0))$, then $|v(t)| \leq A \int_0^t |v(\tau)|\dd\tau$.
\end{lemma}

\begin{proof}
	By \eqref{eq:15} we get
	\begin{equation*}
		\begin{split}
			& \psi(x(u, t)) -\psi(\exp t f_0(x_0)) = \sum_{i\geq 0} L_{f_0}^i \circ L_{f_1} \psi(x_0) \int_0^t \frac{s^i}{i!} u(s) \dd s \\
			& + \sum_{i \geq 1} L_{f_1} \circ L_{f_0}^i \psi\left(x_0\right) \left(I^i \circ I_u\right) \mathfrak{i}(t)+\sum_{|k| \geq 2} D(k, h) \psi\left(x_0\right) p(k, h, u, t) \\
			& =v(t) \sum_{i \geq 0} L_{f_0}^i \circ L_{f_1} \psi\left(x_0\right) \frac{t^i}{i !}-\sum_{i \geq 1} L_{f_0}^i \circ L_{f_1} \psi\left(x_0\right) \int_0^t \frac{s^{i-1}}{(i-1) !} v(s) \dd s+ \text{(the other terms).}
		\end{split}
	\end{equation*}
	Hence by \eqref{eq:13} and \eqref{eq:19} we get
	\begin{equation*}
		\left|v(t)\left(1+\sum_{i \geq 1} L_{f_0}^i \circ L_{f_1} 	\psi\left(x_0\right) \frac{t^i}{i!}\right)\right| 
		\leq
		 \sum_{\substack{|k| \geq 1 \\ \ell=| h|+| k \mid \geq 2}} K_\psi^{\ell} \ell ! 2 \frac{4^\ell}{(\ell-1) !} t^{\ell-2} \int_0^t|v(\tau)| \dd \tau .
	\end{equation*}
	The series $\sum_{i \geq 1} L_{f_0}^i \circ L_{f_1} \psi\left(x_0\right) \frac{t^i}{i!}$ converges to a $C^\omega$ function $\rho(t)$ for $0 \leq t<\frac{1}{K_\psi}$. $\rho(0)=0$, so there is $T_1$ such that $\rho(t)>-\frac{1}{2}$ for each $t \leq T_1$.
	
	We get for each $t \leq T_1$
	\begin{equation*}
		|v(t)| \leq 2 \sum_{\ell \geq 2} 2^{\ell} 2 K_\psi^{\ell} 4^{\ell} \ell t^{\ell-2} \int_0^t|v(\tau)| \dd \tau=4 \sum_{\ell \geq 2}\left(8 K_\psi\right)^{\ell} \ell t^{\ell-2} \int_0^t|v(\tau)| \dd \tau .
	\end{equation*}
	The proof follows as in \cref{lem:3.2}.
\end{proof}

\begin{thebibliography}{99}
	\bibitem{1}
	Fliess, M. --
	``Fonctionnelles causales non linéaires et indéterminées non commutatives'',
	Bull.~Soc.~Math.~France, 109 (1981), 3-40.
	
	\bibitem{2}
	Sussmann, H.J.\ and Jurdjevic, V. --
	``Controllability of non linear systems'',
	J.~Diff.~Eq., 12 (1972), 95-116.
	
	\bibitem{3}
	Sussmann, H.J. --
	``Lie brackets and local controllability: a sufficient condition for scalar-input systems'', 
	SIAM J.~Control Optim, 21 (1983), 686-713.
\end{thebibliography}

\end{document}

